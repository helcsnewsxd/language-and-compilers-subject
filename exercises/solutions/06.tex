\documentclass{article}
\usepackage{geometry}
\usepackage{titling}
\usepackage{hyperref}
\usepackage{amsmath}
\usepackage{amsthm}
\usepackage{amssymb}
\usepackage{graphicx}
\usepackage{caption}
\usepackage{subcaption}
\usepackage{stmaryrd}
\usepackage{enumitem}
\usepackage[dvipsnames]{xcolor}

\geometry{
  a4paper,
  total = {170mm, 257mm},
  left = 20mm,
  top = 20mm,
}
\graphicspath{ {./images/} }

\newcommand{\addfig}[2]{\begin{figure}[!htb] \centering \includegraphics[width=#1\textwidth]{#2}\end{figure}}

\newcommand{\aexp}[1]{\langle\text{#1}\rangle}
\newcommand{\intexp}{\aexp{intexp}}
\newcommand{\var}{\aexp{var}}
\newcommand{\assert}{\aexp{assert}}
\newcommand{\boolexp}{\aexp{boolexp}}
\newcommand{\comm}{\aexp{comm}}
\newcommand{\sem}[1]{\left\llbracket #1\right\rrbracket}

\newcommand{\N}{\mathbb{N}}
\newcommand{\Z}{\mathbb{Z}}
\newcommand{\B}{\mathbb{B}}

\newcommand{\x}{\textbf{x}}
\newcommand{\y}{\textbf{y}}
\newcommand{\z}{\textbf{z}}

\newcommand{\supr}{\bigsqcup\limits}

\newcommand{\concat}{\texttt{ ++ }}

\newcommand{\fleq}{\sqsubseteq}
\newcommand{\cdom}{\Sigma \to \Sigma_\bot}
\newcommand{\cdomf}{\Sigma \to \Sigma_\bot'}
\newcommand{\cdomfo}{\Sigma \to \Omega}
\newcommand{\cfbot}{\bot_{\cdom}}
\newcommand{\cfbotf}{\bot_{\cdomf}}
\newcommand{\cfbotfo}{\bot_{\cdomfo}}
\newcommand{\bbot}{\bot\!\!\!\bot}
\newcommand{\ctrue}{\textbf{true}}
\newcommand{\cfalse}{\textbf{false}}
\newcommand{\cskip}{\textbf{skip}}
\newcommand{\cif}[3]{\textbf{if }#1\textbf{ then }#2\textbf{ else }#3}
\newcommand{\cnewvar}[3]{\textbf{newvar }#1 := #2\textbf{ in }#3}
\newcommand{\cwhile}[2]{\textbf{while }#1\textbf{ do }#2}
\newcommand{\cfor}[4]{\textbf{for }#1 := #2\text{ to }#3\text{ do }#4}
\newcommand{\cfail}{\textbf{fail}}
\newcommand{\ccatch}[2]{\textbf{catchin }#1\textbf{ with }#2}
\newcommand{\cabort}[1]{\langle\textbf{abort}, #1\rangle}
\newcommand{\cout}[1]{\langle #1\rangle}

\title{Trabajo práctico N° 6}
\author{Emanuel Nicolás Herrador}
\date{Mayo 2025}

\makeatletter
\def\@maketitle{%
  \newpage
  \null
  \vskip 1em%
  \begin{center}%
  \let \footnote \thanks
    {\LARGE \@title \par}%
    \vskip 1em%
    {\large \@date}%
  \end{center}%
  \par
  \vskip 1em}
\makeatother

\begin{document}

\maketitle

\noindent\begin{tabular}{@{}ll}
	Estudiante & \theauthor \\
\end{tabular}

\section*{Ejercicio 1}
En este ejercicio se pretende dar un programa para cada posible comportamiento en LIS con fallas y output.
Veamos cada caso posible de forma separada.

Respecto a un programa con \textit{cantidad finita de output y luego divergencia}, podemos considerar:
\begin{equation*}
  \cwhile{\ctrue}{\cskip}
\end{equation*}

Un programa con \textit{cantidad finita de output y luego falla} puede ser:
\begin{equation*}
  \cfail 
\end{equation*}

Un programa con \textit{cantidad finita de output y luego terminación} puede ser:
\begin{equation*}
  \cskip
\end{equation*}

Y, finalmente, un programa con \textit{cantidad infinita de output} puede ser:
\begin{equation*}
  \cwhile{\ctrue}{!1}
\end{equation*}

\section*{Ejercicio 2}
Dado el programa $\cwhile{\x>0}{!\x};c$ se pretende calcular la semántica denotacional para cada uno de los casos dependiendo el programa $c$.
Se verá cada uno por separado.

\subsection*{Item A}
Consideramos $c \equiv \cif{\x>0}{\cskip}{\cfail}$.
Luego, el programa a considerar es:
\begin{equation*}
  \cwhile{\x>0}{(!\x; \cif{\x>0}{\cskip}{\cfail})}
\end{equation*}

En base a eso, veamos la semántica del programa.
Sea $w$ la semántica del while, entonces:
\begin{equation*}
  \begin{aligned}
    &\sem{\cwhile{\x>0}{(!\x; \cif{\x>0}{\cskip}{\cfail})}} \sigma = \\ 
    &= w\sigma \\ 
    &= \begin{cases}
      w_* (\sem{!\x; \cif{\x>0}{\cskip}{\cfail}}\sigma) &\text{ si }\sem{\x>0}\sigma \\ 
      \cout{\sigma} &\text{ si }\neg\sem{\x>0}\sigma
    \end{cases} \\ 
    &= \begin{cases}
      w_* (\sem{\cif{\x>0}{\cskip}{\cfail}}_* (\sem{!\x}\sigma)) &\text{ si }\sigma\x > 0 \\ 
      \cout{\sigma} &\text{ si }\sigma\x \leq 0
    \end{cases} \\ 
    &= \begin{cases}
      w_* (\sem{\cif{\x>0}{\cskip}{\cfail}} \cout{\sigma\x,\ \sigma}) &\text{ si }\sigma\x > 0 \\ 
      \cout{\sigma} &\text{ si }\sigma\x \leq 0
    \end{cases} \\ 
    &= \begin{cases}
      \cout{\sigma\x,\ w_* (\sem{\cif{\x>0}{\cskip}{\cfail}} \sigma)} &\text{ si }\sigma\x > 0 \\ 
      \cout{\sigma} &\text{ si }\sigma\x \leq 0 
    \end{cases} \\ 
    &= \begin{cases}
      \cout{\sigma\x,\ w_* (\sem{\cskip} \sigma)} &\text{ si }\sigma\x > 0 \\ 
      \cout{\sigma} &\text{ si }\sigma\x \leq 0 
    \end{cases} \\ 
    &= \begin{cases}
      \cout{\sigma\x, w \sigma} &\text{ si } \sigma\x > 0 \\ 
      \cout{\sigma} &\text{ si }\sigma\x \leq 0 
    \end{cases} \\ 
    &= \begin{cases}
      \cout{\sigma\x} \concat w\sigma &\text{ si }\sigma\x > 0 \\ 
      \cout{\sigma} &\text{ si }\sigma\x \leq 0
    \end{cases}
  \end{aligned}
\end{equation*}

Ahora, para calcular $w$ vamos a considerar $F \in (\cdomfo) \to (\cdomfo)$ tal que:
\begin{equation*}
  F w \sigma = \begin{cases}
    \cout{\sigma\x} \concat w\sigma &\text{ si }\sigma\x > 0 \\ 
    \cout{\sigma} &\text{ si }\sigma\x \leq 0
  \end{cases}
\end{equation*}

Sabemos que como $w$ es un while, entonces $F$ es una función continua.
Ahora, para calcular la semántica del while podemos usar TMPF de modo que $w = \supr_{i \in \N} F^i \cfbotfo$.
Para ello, entonces, se propone la siguiente caracterización para $F^i \cfbotfo$ con $i \geq 1$:
\begin{equation*}
  F^i \cfbotfo = \sigma \mapsto \begin{cases}
    \cout{\overbrace{\sigma\x, \dots, \sigma\x}^{i \text{ veces}}} &\text{ si } \sigma\x > 0 \\ 
    \cout{\sigma} &\text{ si } \sigma\x \leq 0
  \end{cases}
\end{equation*}

Para demostrarlo, vamos a hacer inducción en $i$.
Veamos primero el caso base $i = 1$:
\begin{equation*}
  \begin{aligned}
    F^1 \cfbotfo &= \sigma \mapsto \begin{cases}
                  \cout{\sigma\x} \concat \cfbotfo\sigma &\text{ si }\sigma\x > 0 \\ 
                  \cout{\sigma} &\text{ si }\sigma\x \leq 0
                \end{cases} \\
                &= \sigma \mapsto \begin{cases}
                  \cout{\sigma\x} &\text{ si }\sigma\x > 0 \\ 
                  \cout{\sigma} &\text{ si }\sigma\x \leq 0
                \end{cases}
  \end{aligned}
\end{equation*}
por lo que se cumple.
Ahora, como HI suponemos que la caracterización vale para $k \in \N_{\geq 1}$ y queremos ver $k+1$:
\begin{equation*}
  \begin{aligned}
    F^{k+1} \cfbotfo &= \sigma \mapsto \begin{cases}
                      \cout{\sigma\x} \concat F^k \cfbotfo \sigma &\text{ si }\sigma\x > 0 \\ 
                      \cout{\sigma} &\text{ si }\sigma\x \leq 0
                    \end{cases} \\ 
                     &= \sigma \mapsto \begin{cases}
                       \cout{\sigma\x} \concat \cout{\overbrace{\sigma\x, \dots, \sigma\x}^{k \text{ veces}}} &\text{ si } \sigma\x > 0 \\ 
                       \cout{\sigma} &\text{ si }\sigma\x \leq 0
                     \end{cases} \\ 
                     &= \sigma \mapsto \begin{cases}
                       \cout{\overbrace{\sigma\x, \dots, \sigma\x}^{k+1 \text{ veces}}} &\text{ si }\sigma\x > 0 \\ 
                       \cout{\sigma} &\text{ si }\sigma\x \leq 0
                     \end{cases}
  \end{aligned}
\end{equation*}

Luego, entonces, por TMPF queda claro que la semántica del while es:
\begin{equation*}
  w = \sigma \mapsto \begin{cases}
    \cout{\overbrace{\sigma\x, \dots, \sigma\x, \dots}^\text{infinitas veces}} &\text{ si }\sigma\x > 0 \\ 
    \cout{\sigma} &\text{ si }\sigma\x \leq 0
  \end{cases}
\end{equation*}

\subsection*{Item B}
Ahora consideramos $c \equiv \cif{\x>0}{\cfail}{\cskip}$.
Por ello, el programa a considerar es:
\begin{equation*}
  \cwhile{\x>0}{(!x; \cif{\x>0}{\cfail}{\cskip})}
\end{equation*}
En base a eso, veamos la semántica del programa.
Sea $w$ la semántica del while, entonces:
\begin{equation*}
  \begin{aligned}
    &\sem{\cwhile{\x>0}{(!x; \cif{\x>0}{\cfail}{\cskip})}} \sigma = \\
    &= w\sigma \\ 
    &= \begin{cases}
      w_* (\sem{!x; \cif{\x>0}{\cfail}{\cskip}}\sigma) &\text{ si }\sem{\x>0}\sigma \\ 
      \cout{\sigma} &\text{ si }\neg\sem{\x>0}\sigma
    \end{cases} \\ 
    &= \begin{cases}
      w_* (\sem{\cif{\x>0}{\cfail}{\cskip}}_* (\sem{!x}\sigma)) &\text{ si }\sigma\x > 0 \\ 
      \cout{\sigma} &\text{ si }\sigma\x \leq 0 
    \end{cases} \\ 
    &= \begin{cases}
      w_* (\sem{\cif{\x>0}{\cfail}{\cskip}}) \cout{\sigma\x, \sigma} &\text{ si }\sigma\x > 0 \\ 
      \cout{\sigma} &\text{ si }\sigma\x \leq 0 
    \end{cases} \\ 
    &= \begin{cases}
      \cout{\sigma\x} \concat w_* (\sem{\cif{\x>0}{\cfail}{\cskip}} \sigma) &\text{ si }\sigma\x > 0 \\ 
      \cout{\sigma} &\text{ si }\sigma\x \leq 0
    \end{cases} 
  \end{aligned}
\end{equation*}

\begin{equation*}
  \begin{aligned}
    &= \begin{cases}
      \cout{\sigma\x} \concat w_* (\sem{\cfail}\sigma) &\text{ si }\sigma\x > 0 \\ 
      \cout{\sigma} &\text{ si }\sigma\x \leq 0
    \end{cases} \\ 
    &= \begin{cases}
      \cout{\sigma\x} \concat w_* \cabort{\sigma} &\text{ si }\sigma\x > 0 \\ 
      \cout{\sigma} &\text{ si }\sigma\x \leq 0
    \end{cases} \\ 
    &= \begin{cases}
      \cout{\sigma\x, \cabort{\sigma}} &\text{ si }\sigma\x > 0 \\ 
      \cout{\sigma} &\text{ si } \sigma\x \leq 0
    \end{cases}
  \end{aligned}
\end{equation*}

En base a esto, entonces, la semántica del while está dada por:
\begin{equation*}
  w = \sigma \mapsto \begin{cases}
    \cout{\sigma\x, \cabort{\sigma}} &\text{ si }\sigma\x > 0 \\ 
    \cout{\sigma} &\text{ si }\sigma\x \leq 0
  \end{cases}
\end{equation*}

\end{document}
