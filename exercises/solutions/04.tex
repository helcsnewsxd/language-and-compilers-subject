\documentclass{article}
\usepackage{geometry}
\usepackage{titling}
\usepackage{hyperref}
\usepackage{amsmath}
\usepackage{amsthm}
\usepackage{amssymb}
\usepackage{graphicx}
\usepackage{caption}
\usepackage{subcaption}
\usepackage{stmaryrd}
\usepackage{enumitem}
\usepackage[dvipsnames]{xcolor}

\geometry{
  a4paper,
  total = {170mm, 257mm},
  left = 20mm,
  top = 20mm,
}
\graphicspath{ {./images/} }

\newcommand{\addfig}[2]{\begin{figure}[!htb] \centering \includegraphics[width=#1\textwidth]{#2}\end{figure}}

\newcommand{\aexp}[1]{\langle\text{#1}\rangle}
\newcommand{\intexp}{\aexp{intexp}}
\newcommand{\var}{\aexp{var}}
\newcommand{\assert}{\aexp{assert}}
\newcommand{\boolexp}{\aexp{boolexp}}
\newcommand{\comm}{\aexp{comm}}
\newcommand{\sem}[1]{\left\llbracket #1\right\rrbracket}

\newcommand{\N}{\mathbb{N}}
\newcommand{\Z}{\mathbb{Z}}
\newcommand{\B}{\mathbb{B}}

\newcommand{\x}{\textbf{x}}
\newcommand{\y}{\textbf{y}}
\newcommand{\z}{\textbf{z}}

\newcommand{\supr}{\bigsqcup\limits}

\newcommand{\fleq}{\sqsubseteq}
\newcommand{\cdom}{\Sigma \to \Sigma_\bot}
\newcommand{\cfbot}{\bot_{\cdom}}
\newcommand{\bbot}{\bot\!\!\!\bot}
\newcommand{\ctrue}{\textbf{true}}
\newcommand{\cfalse}{\textbf{false}}
\newcommand{\cskip}{\textbf{skip}}
\newcommand{\cif}[3]{\textbf{if }#1\textbf{ then }#2\textbf{ else }#3}
\newcommand{\cnewvar}[3]{\textbf{newvar }#1 := #2\textbf{ in }#3}
\newcommand{\cwhile}[2]{\textbf{while }#1\textbf{ do }#2}
\newcommand{\cfor}[4]{\textbf{for }#1 := #2\text{ to }#3\text{ do }#4}

\title{Trabajo práctico N° 4}
\author{Emanuel Nicolás Herrador}
\date{Mayo 2025}

\makeatletter
\def\@maketitle{%
  \newpage
  \null
  \vskip 1em%
  \begin{center}%
  \let \footnote \thanks
    {\LARGE \@title \par}%
    \vskip 1em%
    {\large \@date}%
  \end{center}%
  \par
  \vskip 1em}
\makeatother

\begin{document}

\maketitle

\noindent\begin{tabular}{@{}ll}
	Estudiante & \theauthor \\
\end{tabular}

\section*{Ejercicio 1}
Queremos ver si se demuestran o refutan las equivalencias de comandos que se proponen en los siguientes items.

\subsection*{Item A}
Queremos ver que $c; \cskip \equiv c$.
Sea $\sigma \in \Sigma$, se cumple que:
\begin{equation*}
  \begin{aligned}
    \sem{c; \cskip}\sigma &= \sem{\cskip}_{\bbot}(\sem{c}\sigma) \\ 
                          &= \begin{cases}
                            \bot &\text{ si }\sem{c}\sigma = \bot \\ 
                            \sem{skip}(\sem{c}\sigma) &\text{ cc. }
                          \end{cases} \\ 
                          &= \begin{cases}
                            \bot &\text{ si }\sem{c}\sigma = \bot \\ 
                            \sem{c}\sigma &\text{ cc. }
                          \end{cases} \\ 
                          &= \sem{c}\sigma 
  \end{aligned}
\end{equation*}
por lo que se demuestra la equivalencia propuesta. $\qed$

\subsection*{Item B}
Queremos ver si $c_1; (c_2; c_3) \equiv (c_1; c_2); c_3$.
Sea $\sigma \in \Sigma$ notemos que:
\begin{equation*}
  \begin{aligned}
    \sem{c1; (c_2; c_3)}\sigma &= \sem{c_2; c_3}_{\bbot}(\sem{c_1}\sigma) \\ 
                               &= \sem{c_3}_{\bbot}(\sem{c_2}_{\bbot}(\sem{c_1}\sigma)) \\ 
                               &= \sem{c_3}_{\bbot}(\sem{c1; c2}\sigma) \\ 
                               &= \sem{(c_1; c_2); c_3}\sigma
  \end{aligned}
\end{equation*}
por lo que se demuestra la equivalencia propuesta. $\qed$.

\subsection*{Item C}
Queremos ver si $(\cif{b}{c_0}{c_1});c_2 \equiv \cif{b}{c_0;c_2}{c_1;c_2}$.
Sea $\sigma \in \Sigma$, notemos que:
\begin{equation*}
  \begin{aligned}
    \sem{(\cif{b}{c_0}{c_1});c_2}\sigma &= \sem{c_2}_{\bbot}(\sem{\cif{b}{c_0}{c_1}}\sigma) \\ 
                                        &= \begin{cases}
                                          \sem{c_2}_{\bbot}(\sem{c_0}\sigma) &\text{ si }\sem{b}\sigma \\ 
                                          \sem{c_2}_{\bbot}(\sem{c_1}\sigma) &\text{ cc. }
                                        \end{cases} \\ 
                                        &= \begin{cases}
                                          \sem{c_0; c_2}\sigma & \text{ si }\sem{b}\sigma \\ 
                                          \sem{c_1; c_2}\sigma & \text{ cc. }
                                        \end{cases} \\ 
                                        &= \sem{\cif{b}{c_0; c_2}{c_1; c_2}}\sigma
  \end{aligned}
\end{equation*}
por lo que se demuestra la equivalencia propuesta. $\qed$

\subsection*{Item D}
Queremos ver si $c_2; (\cif{b}{c_0}{c_1}) \equiv \cif{b}{c_2; c_0}{c_2; c_1}$.
Sea $\sigma \in \Sigma$, veamos que:
\begin{equation*}
  \begin{aligned}
    \sem{c_2; (\cif{b}{c_0}{c_1})}\sigma &= \sem{(\cif{b}{c_0}{c_1})}_{\bbot}(\sem{c_2}\sigma) \\ 
                                         &= \begin{cases}
                                           \bot & \text{ si }\sem{c_2}\sigma = \bot \\ 
                                           \sem{c_0}(\sem{c_2}\sigma) & \text{ si }\sem{c_2}\sigma \neq \bot \land \sem{b}(\sem{c_2}\sigma) \\ 
                                           \sem{c_1}(\sem{c_2}\sigma) & \text{ si }\sem{c_2}\sigma \neq \bot \land \neg\sem{b}(\sem{c_2}\sigma)
                                         \end{cases} \\ 
                                         \\ 
    \sem{\cif{b}{c_2;c_0}{c_2;c_1}}\sigma &= \begin{cases}
                                      \sem{c_2; c_0}\sigma &\text{ si }\sem{b}\sigma \\ 
                                      \sem{c_2; c_1}\sigma &\text{ cc. }
                                    \end{cases} \\ 
                                  &= \begin{cases}
                                      \sem{c_0}_{\bbot}(\sem{c_2}\sigma) & \text{ si }\sem{b}\sigma \\ 
                                      \sem{c_1}_{\bbot}(\sem{c_2}\sigma) & \text{ cc. }
                                    \end{cases} \\ 
                                  &= \begin{cases}
                                    \bot & \text{ si }\sem{c_2}\sigma = \bot \\ 
                                    \sem{c_0}(\sem{c_2}\sigma) & \text{ si }\sem{c_2}\sigma \neq \bot \land \sem{b}\sigma \\ 
                                    \sem{c_1}(\sem{c_2}\sigma) & \text{ si }\sem{c_2}\sigma \neq \bot \land \neg\sem{b}\sigma
                                  \end{cases}
  \end{aligned}
\end{equation*}

Con ello, entonces, puede observarse que no son equivalentes debido a que los estados en los que se evalúa la condición $b$ pueden ser distintos.

\subsection*{Item E}
Queremos ver si $x := y; z := w \equiv z := w; x := y$.
Si bien es claro que no se va a cumplir esta equivalencia, lo veamos por la semántica.
Sea $\sigma \in \Sigma$, veamos que:
\begin{equation*}
  \begin{aligned}
    \sem{x := y; z := w}\sigma &= \sem{z := w}_{\bbot}(\sem{x := y}\sigma) \\ 
                               &= \sem{z := w}_{\bbot}[\sigma | x : \sem{y}\sigma] \\ 
                               &= \sem{z := w}[\sigma | x : y] \\ 
                               &= [\sigma | x : y | z : \sem{w}\sigma] \\ 
                               &= [\sigma | x : y | z : w] \\ 
                               \\ 
    \sem{z := w; x := y}\sigma &= \sem{x := y}_{\bbot}(\sem{z := w}\sigma) \\ 
                               &= \sem{x := y}_{\bbot}[\sigma | z : \sem{w}\sigma] \\ 
                               &= \sem{x := y}[\sigma | z : w] \\ 
                               &= [\sigma | z : w | x : \sem{y}\sigma] \\ 
                               &= [\sigma | z : w | x : y]
  \end{aligned}
\end{equation*}

Con ello, entonces, puede observarse que no son equivalentes debido a que en este ejercicio al no estar en negrita y letra uniforme, no se tratan de variables sino de metavariables.
Luego, esto significa que si consideramos una asignación de variables dada por $x \to \textbf{x},\ y \to \textbf{y},\ z \to \textbf{x},\ w \to \textbf{w}$ entonces es claro que los estados resultantes son diferentes.
Esto es debido a que $[\sigma | \textbf{x} : \textbf{y} | \textbf{x} : \textbf{w}] = [\sigma | \textbf{x} : \textbf{w}]$ pero $[\sigma | \textbf{x} : \textbf{w} | \textbf{x} : \textbf{y}] = [\sigma | \textbf{x} : \textbf{y}]$.

\section*{Ejercicio 2}
El ejercicio es similar al anterior, pero ahora con $\textbf{newvar}$.

\subsection*{Item A}
Queremos ver si es correcta la equivalencia $\cnewvar{\x}{e}{\cskip} \equiv \cskip$.
Sea $\sigma \in \Sigma$, veamos que:
\begin{equation*}
  \begin{aligned}
    \cnewvar{\x}{e}{\cskip} &= (\lambda \sigma' \in \Sigma . [\sigma' | \x : \sigma \x])_{\bbot}(\sem{\cskip}[\sigma | \x : \sem{e}\sigma]) \\ 
                            &= (\lambda \sigma' \in \Sigma . [\sigma' | \x : \sigma \x][\sigma | \x : \sem{e}\sigma]) \\ 
                            &= [\sigma | \x : \sem{e}\sigma | \x : \sigma \x] \\ 
                            &= \sigma \\ 
                            &= \sem{\cskip}\sigma
  \end{aligned}
\end{equation*}
por lo que, entonces, se demuestra la equivalencia. $\qed$

\subsection*{Item B}
Queremos ver si es correcta la equivalencia $\cnewvar{\x}{e}{\y := \x} \equiv \y := e$.
Sea $\sigma \in \Sigma$, veamos que:
\begin{equation*}
  \begin{aligned}
    \sem{\cnewvar{\x}{e}{\y := \x}}\sigma &= (\lambda \sigma' \in \Sigma . [\sigma' | \x : \sigma \x])_{\bbot}(\sem{\y := \x}[\sigma | \x : \sem{e}\sigma]) \\ 
                                          &= (\lambda \sigma' \in \Sigma . [\sigma' | \x : \sigma \x])[\sigma | \x : \sem{e}\sigma | \y : \sem{\x}[\sigma | \x : \sem{e}\sigma]] \\ 
                                          &= (\lambda \sigma' \in \Sigma . [\sigma' | \x : \sigma \x])[\sigma | \x : \sem{e}\sigma | \y : \sem{e}\sigma] \\
                                          &= [\sigma | \x : \sem{e}\sigma | \y : \sem{e}\sigma | \x : \sigma \x] \\ 
                                          &= [\sigma | \y : \sem{e}\sigma] \\ 
                                          &= \sem{\y := e}\sigma
  \end{aligned}
\end{equation*}
por lo que, entonces, se demuestra la equivalencia. $\qed$

\subsection*{Item C}
Queremos ver si es correcta la siguiente equivalencia
\begin{equation*}
  \cnewvar{\x}{e_1}{(\cnewvar{\y}{e_2}{c})} \equiv \cnewvar{\y}{e_2}{(\cnewvar{\x}{e_1}{c})}
\end{equation*}

Sea $\sigma \in \Sigma$, veamos que:
\begin{equation*}
  \begin{aligned}
    &\sem{\cnewvar{\x}{e_1}{(\cnewvar{\y}{e_2}{c})}}\sigma = \\ 
    &= (\lambda \sigma' \in \Sigma . [\sigma'\ |\ \x : \sigma \x])_{\bbot}(\sem{\cnewvar{\y}{e_2}{c}}[\sigma\ |\ \x : \sem{e_1}\sigma]) \\ 
    &= (\lambda \sigma' \in \Sigma . [\sigma'\ |\ \x : \sigma \x])_{\bbot}\left((\lambda \sigma'' \in \Sigma . [\sigma''\ |\ \y : [\sigma\ |\ \x : \sem{e_1}\sigma] \y])_{\bbot} \sem{c}[\sigma\ |\ \x : \sem{e_1}\sigma\ |\ \y : \sem{e_2}[\sigma\ |\ \x : \sem{e_1}\sigma]]\right) \\ 
    &= (\lambda \sigma' \in \Sigma . [\sigma'\ |\ \x : \sigma \x\ |\ \y : \sigma \y])_{\bbot} (\sem{c}[\sigma\ |\ \x : \sem{e_1}\sigma\ |\ \y : \sem{e_2}[\sigma\ |\ \x : \sem{e_1}\sigma]]) \\ 
    \\ 
    &\sem{\cnewvar{\y}{e_2}{(\cnewvar{\x}{e_1}{c})}}\sigma = \\ 
    &= (\lambda \sigma' \in \Sigma . [\sigma'\ |\ \y : \sigma \y])_{\bbot}(\sem{\cnewvar{\x}{e_1}{c}}[\sigma\ |\ \y : \sem{e_2}\sigma]) \\ 
    &= (\lambda \sigma' \in \Sigma . [\sigma'\ |\ \y : \sigma \y])_{\bbot}\left((\lambda \sigma'' \in \Sigma . [\sigma''\ |\ \x : [\sigma\ |\ \y : \sem{e_2}\sigma] \x])_{\bbot} \sem{c}[\sigma\ |\ \y : \sem{e_2}\sigma\ |\ \x : \sem{e_1}[\sigma\ |\ \y : \sem{e_2}\sigma]]\right) \\ 
    &= (\lambda \sigma' \in \Sigma . [\sigma'\ |\ \y : \sigma \y\ |\ \x : \sigma \x])_{\bbot} (\sem{c}[\sigma\ |\ \y : \sem{e_2}\sigma\ |\ \x : \sem{e_1}[\sigma\ |\ \y : \sem{e_2}\sigma]]) \\ 
    &= (\lambda \sigma' \in \Sigma . [\sigma'\ |\ \x : \sigma \x\ |\ \y : \sigma \y])_{\bbot} (\sem{c}[\sigma\ |\ \x : \sem{e_1}[\sigma\ |\ \y : \sem{e_2}\sigma]\ |\ \y : \sem{e_2}\sigma])
  \end{aligned}
\end{equation*}

Luego, claramente no son equivalentes y esto lo podemos mostrar presentando un contraejemplo de ello.
Consideremos $c = \z := \y,\ e_1 = \x + 1,\ e_2 = \x$.
Con ello, tenemos para cada caso que:
\begin{equation*}
  \begin{aligned}
    &\sem{\cnewvar{\x}{\x + 1}{(\cnewvar{\y}{\x}{\z := \y})}}\sigma = \\ 
    &= (\lambda \sigma' \in \Sigma . [\sigma'\ |\ \x : \sigma \x\ |\ \y : \sigma \y])_{\bbot} (\sem{\z := \y}[\sigma\ |\ \x : \sem{\x + 1}\sigma\ |\ \y : \sem{\x}[\sigma | \x : \sem{\x + 1}\sigma]]) \\ 
    &= (\lambda \sigma' \in \Sigma . [\sigma'\ |\ \x : \sigma \x\ |\ \y : \sigma \y])_{\bbot} (\sem{\z := \y}[\sigma\ |\ \x : \sem{\x + 1}\sigma\ |\ \y : \sem{\x + 1}\sigma]) \\ 
    &= (\lambda \sigma' \in \Sigma . [\sigma'\ |\ \x : \sigma \x\ |\ \y : \sigma \y])_{\bbot} [\sigma\ |\ \x : \sem{\x + 1}\sigma\ |\ \y : \sem{\x + 1}\sigma\ |\ \z : [\sigma\ |\ \x : \sem{\x + 1}\sigma\ |\ \y : \sem{\x + 1}\sigma] \y] \\
    &= (\lambda \sigma' \in \Sigma . [\sigma'\ |\ \x : \sigma \x\ |\ \y : \sigma \y]) [\sigma\ |\ \x : \sem{\x + 1}\sigma\ |\ \y : \sem{\x + 1}\sigma\ |\ \z : \sem{\x + 1}\sigma] \\ 
    &= [\sigma\ |\ \z : \sem{\x + 1}\sigma] \\ 
    \\ 
    &\sem{\cnewvar{\y}{\x}{(\cnewvar{\x}{\x + 1}{\z := \y})}}\sigma = \\ 
    &= (\lambda \sigma' \in \Sigma . [\sigma'\ |\ \x : \sigma \x\ |\ \y : \sigma \y])_{\bbot} (\sem{\z := \y}[\sigma\ |\ \x : \sem{\x + 1}[\sigma\ |\ \y : \sem{\x}\sigma]\ |\ \y : \sem{\x}\sigma]) \\ 
    &= (\lambda \sigma' \in \Sigma . [\sigma'\ |\ \x : \sigma \x\ |\ \y : \sigma \y])_{\bbot} (\sem{\z := \y}[\sigma\ |\ \x : \sem{\x + 1}\sigma\ |\ \y : \sem{\x}\sigma]) \\ 
    &= (\lambda \sigma' \in \Sigma . [\sigma'\ |\ \x : \sigma \x\ |\ \y : \sigma \y]) [\sigma\ |\ \x : \sem{\x + 1}\sigma\ |\ \y : \sem{\x}\sigma\ |\ \z : [\sigma\ |\ \x : \sem{\x + 1}\sigma\ |\ \y : \sem{\x}\sigma] \y]) \\ 
    &= (\lambda \sigma' \in \Sigma . [\sigma'\ |\ \x : \sigma \x\ |\ \y : \sigma \y]) [\sigma\ |\ \x : \sem{\x + 1}\sigma\ |\ \y : \sem{\x}\sigma\ |\ \z : \sem{x}\sigma]) \\ 
    &= [\sigma\ |\ \z : \sem{x}\sigma]
  \end{aligned}
\end{equation*}
por lo que se llegan a estados distintos, lo que significa que entonces no se cumple la equivalencia.
Por ello, se refuta.

\section*{Ejercicio 3}
Respecto al punto $(a)$, considero que el parser \textit{no puede} eliminar toda ocurrencia de $\cskip$ porque ello ``rompería'' nuestra sintaxis.
Un ejemplo de esto es que en el ejercicio $(2a)$ tendríamos $\cnewvar{\x}{e}{}$ lo cual incumple nuestra definición de LIS.

Y, respecto al punto $(b)$, \textit{sí es posible} la elección de cómo asociar las secuencias de comandos.
Esto se demostró en el ejercicio $(1b)$.

\section*{Ejercicio 4}
Para este ejercicio vamos a considerar el comando $\cwhile{\ctrue}{x := x-1}$.
Veamos cada uno de los items a continuación.

\subsection*{Item A}
Sea $w$ la función que representa la semántica del while, notemos que $w : \Sigma \to \Sigma_\bot$ y que:
\begin{equation*}
  \begin{aligned}
      w \sigma &= \begin{cases}
              w_{\bbot} (\sem{x := x-1}\sigma) &\text{ si }\sem{\ctrue}\sigma \\ 
              \sigma & \text{ cc. }
            \end{cases} \\ 
               &= w [\sigma\ |\ x : \sem{x}\sigma - 1] \\ 
               &= w [\sigma | x : \sigma x - 1]
  \end{aligned}
\end{equation*}

Con ello, entonces, se puede definir $F : (\Sigma \to \Sigma_\bot) \to (\Sigma \to \Sigma_\bot)$ tal que:
\begin{equation*}
  \begin{aligned}
    F w \sigma &= w [\sigma | x : \sigma x - 1]
  \end{aligned}
\end{equation*}

\subsection*{Item B}
Ahora calculemos primero $F^i \cfbot$ para algunos $i \in \N$:
\begin{equation*}
  \begin{aligned}
    F^0 \cfbot &= \cfbot \\ 
    \\ 
    F^1 \cfbot &= \sigma \mapsto \cfbot [\sigma\ |\ x : \sigma x - 1] \\ 
               &= \sigma \mapsto \bot \\ 
               &= \cfbot
  \end{aligned}
\end{equation*}

Con ello, puede observarse que claramente $F^i \cfbot = \cfbot$.
Vamos a demostrarlo por inducción.
El caso base ya está probado, ahora si suponemos que vale para $k \in \N$, queremos verlo para $k+1$:
\begin{equation*}
  \begin{aligned}
    F^{k+1} \cfbot &= F F^k \cfbot \\ 
                   &= F \cfbot \\ 
                   &= \cfbot
  \end{aligned}
\end{equation*}

Luego, se demuestra que $\forall i \in \N,\ F^i \cfbot = \cfbot$.
Es decir, no existe ningún $n$ tal que $F^n \cfbot$ no sea bottom.

\subsection*{Item C}
Consideramos la cadena en $\cdom$ dada por:
\begin{equation*}
  w_i \sigma = \begin{cases}
    \sigma &\text{ si }0\leq \sigma x \leq i \\ 
    \bot & \text{ cc. }
  \end{cases}
\end{equation*}

Con ello, queremos comprobar que dado que $F$ es continua entonces se garantiza $F(\supr w_i) = \supr F w_i$.
Primero veamos el lado izquierdo de la igualdad:
\begin{equation*}
    \begin{aligned}
      \supr w_i &= \sigma \mapsto \begin{cases}
        \sigma &\text{ si }\sigma x \in \N \\ 
        \bot &\text{ cc. }
      \end{cases} \\ 
      \\ 
        F\left(\supr w_i\right) &= \sigma \mapsto \left(\supr w_i\right)[\sigma\ |\ x : \sigma x - 1] \\ 
                     &= \sigma \mapsto \begin{cases}
                       [\sigma\ |\ x : \sigma x - 1] &\text{ si }[\sigma\ |\ x : \sigma x - 1] x \in \N \\ 
                       \bot &\text{ cc. }
                     \end{cases} \\ 
                     &= \sigma \mapsto \begin{cases}
                       [\sigma\ |\ x : \sigma x - 1] &\text{ si }\sigma x \geq 1 \\ 
                       \bot &\text{ cc. }
                     \end{cases}
    \end{aligned}
\end{equation*}

Ahora, veamos el lado derecho de la igualdad:
\begin{equation*}
  \begin{aligned}
    F w_i &= \sigma \mapsto w_i[\sigma\ |\ x : \sigma x - 1] \\ 
          &= \sigma \mapsto \begin{cases}
            [\sigma\ |\ x : \sigma x - 1] &\text{ si }0 \leq [\sigma\ |\ x : \sigma x - 1] x \leq i \\ 
            \bot & \text{ cc. }
          \end{cases} \\ 
          &= \sigma \mapsto \begin{cases}
            [\sigma\ |\ x : \sigma x - 1] &\text{ si }1 \leq \sigma x \leq i + 1 \\ 
            \bot &\text{ cc. }
          \end{cases} \\
          \\ 
    \supr F w_i &= \sigma \mapsto \begin{cases}
      [\sigma\ |\ x : \sigma x - 1] &\text{ si }1 \leq \sigma x \\ 
      \bot &\text{ cc. }
    \end{cases} \\ 
  \end{aligned}
\end{equation*}

Luego, de este modo, se corrobora la igualdad y, por ende, la conservación del supremo en la cadena $w_0 \leq w_1 \leq \dots$.

\section*{Ejercicio 5}
Se pretende calcular la semántica denotacional de los comandos presentados a continuación.

\subsection*{Item A}
Se considera el comando $\cwhile{\x < 2}{\cif{\x < 0}{\x := 0}{\x := \x + 1}}$.
Sea $w \in \cdom$ la función semántica para el while, notemos que:
\begin{equation*}
  \begin{aligned}
    w \sigma &= \begin{cases}
      w_{\bbot} (\sem{\cif{\x<0}{\x:=0}{\x:=\x+1}}\sigma) &\text{ si }\sem{\x<2}\sigma \\ 
      \sigma &\text{ cc. }
    \end{cases} \\ 
             &= \begin{cases}
               w_{\bbot}(\sem{\x:=0}\sigma) &\text{ si }\sem{\x<2}\sigma \land \sem{\x<0}\sigma \\ 
               w_{\bbot}(\sem{\x:=\x+1}\sigma) &\text{ si }\sem{\x<2}\sigma \land \neg\sem{\x<0}\sigma \\ 
               \sigma &\text{ cc. }
             \end{cases} \\ 
             &= \begin{cases}
               w [\sigma\ |\ \x : 0] &\text{ si }\sigma \x < 0 \\ 
               w [\sigma\ |\ \x : \sigma x + 1] &\text{ si }0 \leq \sigma \x < 2 \\ 
               \sigma &\text{ cc. }
             \end{cases}
  \end{aligned}
\end{equation*}

Sea $F \in (\cdom) \to (\cdom)$, consideramos:
\begin{equation*}
  F w \sigma = \begin{cases}
    w [\sigma\ |\ \x : 0] &\text{ si }\sigma \x < 0 \\ 
    w [\sigma\ |\ \x : \sigma \x + 1] &\text{ si }0 \leq \sigma \x < 2 \\ 
    \sigma &\text{ si }2 \leq \sigma \x
  \end{cases}
\end{equation*}

Con ello en mente, sabemos que $F$ es continua al ser la semántica de un while por lo que podemos aplicar directamente el TMPF.
Veamos algunos casos para $F^i \cfbot$:
\begin{equation*}
  \begin{aligned}
    F^0 \cfbot &= \cfbot \\ 
    \\ 
    F^1 \cfbot &= \sigma \mapsto \begin{cases}
      \cfbot [\sigma\ |\ \x : 0] &\text{ si }\sigma \x < 0 \\ 
      \cfbot [\sigma\ |\ \x : \sigma \x + 1] &\text{ si }0 \leq \sigma \x < 2 \\ 
      \sigma &\text{ si }2 \leq \sigma \x 
    \end{cases} \\ 
               &= \sigma \mapsto \begin{cases}
                 \bot &\text{ si }\sigma \x < 2 \\ 
                 \sigma &\text{ si }2 \leq \sigma \x
               \end{cases} \\ 
  \end{aligned}
\end{equation*}

\begin{equation*}
  \begin{aligned}
      F^2 \cfbot &= F F \cfbot \\ 
                 &= \sigma \mapsto \begin{cases}
                   F \cfbot [\sigma\ |\ \x : 0] &\text{ si }\sigma \x < 0 \\ 
                   F \cfbot [\sigma\ |\ \x : \sigma \x + 1] &\text{ si }0 \leq \sigma \x < 2 \\ 
                   \sigma &\text{ si }2 \leq \sigma \x
                 \end{cases} \\ 
                 &= \sigma \mapsto \begin{cases}
                   \bot &\text{ si }\sigma \x < 0 \land 0 < 2 \\ 
                   [\sigma\ |\ \x : 0] &\text{ si }\sigma \x < 0 \land 2 \leq 0 \\ 
                   \bot &\text{ si }0 \leq \sigma \x < 2 \land \sigma \x + 1 < 2 \\ 
                   [\sigma\ |\ \x : \sigma \x + 1] &\text{ si } 0 \leq \sigma \x < 2 \land 2 \leq \sigma \x + 1 \\ 
                   \sigma &\text{ si }2 \leq \sigma \x
                 \end{cases} \\ 
                 &= \sigma \mapsto \begin{cases}
                   \bot &\text{ si }\sigma \x \leq 0 \\ 
                   [\sigma\ |\ \x : \sigma \x + 1] &\text{ si }\sigma \x = 1 \\ 
                   \sigma &\text{ si }2 \leq \sigma \x 
                 \end{cases} \\ 
                 &= \sigma \mapsto \begin{cases}
                   \bot &\text{ si }\sigma \x \leq 0 \\ 
                   [\sigma\ |\ \x : \max(2, \sigma \x)] &\text{ si }1 \leq \sigma \x
                 \end{cases} \\ 
                 \\ 
      F^3 \cfbot &= F F^2 \cfbot \\ 
                 &= \sigma \mapsto \begin{cases}
                   F^2 \cfbot [\sigma\ |\ \x : 0] &\text{ si }\sigma \x < 0 \\ 
                   F^2 \cfbot [\sigma\ |\ \x : \sigma \x + 1] &\text{ si }0 \leq \sigma x < 2 \\ 
                   \sigma &\text{ si }2 \leq \sigma \x
                 \end{cases} \\ 
                 &= \sigma \mapsto \begin{cases}
                   \bot &\text{ si }\sigma \x < 0 \land 0 \leq 0 \\ 
                   [\sigma\ |\ \x : 0\ |\ \x : \max(2, 0)] &\text{ si }1 \leq 0 \\ 
                   \bot &\text{ si } 0 \leq \sigma \x < 2 \land \sigma \x + 1 \leq 0 \\ 
                   [\sigma\ |\ \x : \sigma \x + 1\ |\ \x : \max(2, \sigma \x + 1)] &\text{ si }0 \leq \sigma \x < 2 \land 1 \leq \sigma \x + 1 \\ 
                   \sigma &\text{ si }2 \leq \sigma \x
                 \end{cases} \\ 
                 &= \sigma \mapsto \begin{cases}
                   \bot &\text{ si }\sigma \x < 0 \\ 
                   [\sigma\ |\ \x : \max(2, \sigma \x + 1)] &\text{ si }0 \leq \sigma \x < 2 \\ 
                   \sigma &\text{ si }2 \leq \sigma \x
                 \end{cases} \\ 
                 &= \sigma \mapsto \begin{cases}
                   \bot &\text{ si } \sigma \x < 0 \\ 
                   [\sigma\ |\ \x : \max(2, \sigma \x)] &\text{ si }0 \leq \sigma \x
                 \end{cases} \\ 
                 \\ 
      F^4 \cfbot &= \sigma \mapsto \begin{cases}
        F^3 \cfbot [\sigma\ |\ \x : 0]&\text{ si }\sigma \x < 0 \\ 
        F^3 \cfbot [\sigma\ |\ \x:\sigma\x+1]&\text{ si }0\leq\sigma\x<2 \\ 
        \sigma&\text{ si }2\leq\sigma\x
      \end{cases} \\ 
                 &= \sigma \mapsto \begin{cases}
                   \bot&\text{ si }\sigma\x<0 \land 0<0 \\ 
                   [\sigma\ |\ \x : 0\ |\ \x : \max(2, \sigma\x)] &\text{ si } \sigma\x < 0 \land 0 \leq 0 \\ 
                   \bot &\text{ si }0 \leq \sigma\x < 2 \land \sigma\x+1 < 0 \\ 
                   [\sigma\ |\ \x : \sigma\x+1\ |\ \x : \max(2,\sigma\x+1)] &\text{ si }0 \leq \sigma\x < 2 \land 0 \leq \sigma\x+1 \\ 
                   \sigma &\text{ si }2 \leq \sigma\x
                 \end{cases} \\ 
                 &= \sigma \mapsto \begin{cases}
                   [\sigma\ |\ \x : \max(2, \sigma \x)] &\text{ si }\sigma \x < 0 \\ 
                   [\sigma\ |\ \x : \max(2, \sigma\x+1)] &\text{ si }0 \leq \sigma\x < 2 \\ 
                   \sigma &\text{ si }2 \leq \sigma\x 
                 \end{cases} \\ 
                 &= \sigma \mapsto [\sigma\ |\ \x : \max(2, \sigma\x)]
  \end{aligned}
\end{equation*}

Como en este último llegamos a una definición total para $F^4 \cfbot$, significa que el while realiza un máximo de $3$ iteraciones.
Al ser una cadena y cumplirse que $F^0 \cfbot \fleq F^1 \cfbot \fleq F^2 \cfbot \fleq \dots$, como $F^4 \cfbot$ está definido en todos los valores (i.e., no tiene bottom en la imagen), entonces este es el supremo de la cadena.
Motivo de ello, entonces, tenemos que:
\begin{equation*}
  \supr_{i \in \N} F^i \cfbot = F^4 \cfbot = \sigma \mapsto [\sigma\ |\ \x : \max(2, \sigma\x)]
\end{equation*}

\subsection*{Item B}
Se considera el comando $\cwhile{\x < 2}{\cif{\y = 0}{\x := \x + 1}{\cskip}}$.
Sea $w \in \cfbot$ la función semántica para el while, notemos que:
\begin{equation*}
  \begin{aligned}
    w \sigma &= \begin{cases}
      w_{\bbot} (\sem{\cif{\y = 0}{\x := \x + 1}{\cskip}} \sigma) &\text{ si }\sem{\x < 2}\sigma \\ 
      \sigma &\text{ si }\neg\sem{\x < 2}\sigma
    \end{cases} \\ 
             &= \begin{cases}
               w_{\bbot} (\sem{\x := \x + 1} \sigma) & \text{ si } \sem{\x < 2}\sigma \land \sem{\y = 0}\sigma \\ 
               w_{\bbot} (\sem{\cskip} \sigma) &\text{ si }\sem{\x < 2}\sigma \land \neg\sem{\y = 0}\sigma \\ 
               \sigma &\text{ si } \neg\sem{\x < 2}\sigma
             \end{cases} \\ 
             &= \begin{cases}
               w [\sigma\ |\ \x : \sigma\x+1] &\text{ si } \sigma\x < 2 \land \sigma\y = 0 \\ 
               w \sigma &\text{ si } \sigma\x < 2 \land \sigma\y \neq 0 \\ 
               \sigma &\text{ si } 2 \leq \sigma\x
             \end{cases}
  \end{aligned}
\end{equation*}

Sea $F \in (\cdom) \to (\cdom)$, consideramos:
\begin{equation*}
  F w \sigma = \begin{cases}
    w [\sigma\ |\ \x : \sigma\x+1] &\text{ si }\sigma\x < 2 \land \sigma\y = 0 \\ 
    w \sigma &\text{ si }\sigma\x < 2 \land \sigma\y \neq 0 \\ 
    \sigma &\text{ si } 2 \leq \sigma\x
  \end{cases}
\end{equation*}

Con ello en mente, como $F$ es continua al ser la semántica de un while, podemos aplicar directamente el TMPF.
Veamos algunos casos para $F^i \cfbot$:
\begin{equation*}
  \begin{aligned}
    F^0 \cfbot &= \cfbot \\ 
    \\ 
    F^1 \cfbot &= \sigma \mapsto \begin{cases}
      \cfbot [\sigma\ |\ \x : \sigma\x + 1] &\text{ si } \sigma\x < 2 \land \sigma\y = 0 \\ 
      \cfbot \sigma &\text{ si }\sigma\x < 2 \land \sigma\y \neq 0 \\ 
      \sigma &\text{ si }2 \leq \sigma\x
    \end{cases} \\ 
               &= \sigma \mapsto \begin{cases}
                 \bot &\text{ si } \sigma\x < 2 \\ 
                 \sigma &\text{ si } 2 \leq \sigma\x
               \end{cases} \\ 
               \\ 
      F^2 \cfbot &= \sigma \mapsto \begin{cases}
        F \cfbot [\sigma\ |\ \x : \sigma\x+1] &\text{ si }\sigma\x < 2 \land \sigma\y = 0 \\ 
        F \cfbot \sigma &\text{ si }\sigma\x < 2 \land \sigma\y \neq 0 \\ 
        \sigma &\text{ si }2 \leq \sigma\x
      \end{cases} \\ 
                 &= \sigma \mapsto \begin{cases}
                   \bot &\text{ si }\sigma\x < 2 \land \sigma\y = 0 \land \sigma\x+1 < 2 \\ 
                   [\sigma\ |\ \x : \sigma\x+1] &\text{ si }\sigma\x < 2 \land \sigma\y = 0 \land 2 \leq \sigma\x+1 \\ 
                   \bot &\text{ si }\sigma\x < 2 \land \sigma\y \neq 0 \land \sigma\x < 2 \\ 
                   \sigma &\text{ si }\sigma\x < 2 \land \sigma\y \neq 0 \land 2 \leq \sigma\x \\ 
                   \sigma &\text{ si } 2 \leq \sigma\x
                 \end{cases} \\ 
                 &= \sigma \mapsto \begin{cases}
                   \bot &\text{ si }\sigma\x < 1 \land \sigma\y = 0 \\ 
                   [\sigma\ |\ \x : \sigma\x+1] &\text{ si }\sigma\x = 1 \land \sigma\y = 0 \\ 
                   \bot &\text{ si }\sigma\x < 2 \land \sigma\y \neq 0 \\ 
                   \sigma &\text{ si } 2 \leq \sigma\x
                 \end{cases} \\ 
                 &= \sigma \mapsto \begin{cases}
                   \sigma &\text{ si } 2 \leq \sigma\x \\ 
                   [\sigma\ |\ \x : 2] &\text{ si } \sigma\x = 1 \land \sigma\y = 0 \\ 
                   \bot &\text{ si } \sigma\x < 1 \land \sigma\y = 0 \\
                   \bot &\text{ si } \sigma\x < 2 \land \sigma\y \neq 0
                 \end{cases}
  \end{aligned}
\end{equation*}

Viendo esto, como sabemos que $F^0 \cfbot \fleq F^1 \cfbot \fleq \dots$ y buscamos el supremo $\supr_{i \in \N} F^i \cfbot$, entonces es claro que este mismo es igual a $\supr_{i = 1}^\infty F^i \cfbot$.
Esto se menciona porque se propone la siguiente caracterización de $F^i \cfbot$ para $i \in \N_{\geq 1}$:
\begin{equation*}
  F^i \cfbot = \sigma \mapsto \begin{cases}
    \sigma &\text{ si } 2 \leq \sigma\x \\ 
    [\sigma\ |\ \x : 2] &\text{ si } 3-i \leq \sigma\x < 2 \land \sigma\y = 0 \\ 
    \bot &\text{ si } \sigma\x < 3-i \land \sigma\y = 0 \\ 
    \bot &\text{ si } \sigma\x < 2 \land \sigma\y \neq 0 
  \end{cases}
\end{equation*}

Demostremos por inducción que esta caracterización es correcta.
Para el caso base $n = 1$ tenemos que, por lo visto antes, este es igual a:
\begin{equation*}
  F^1 \cfbot = \sigma \mapsto \begin{cases}
    \sigma &\text{ si } 2 \leq \sigma\x \\ 
    \bot &\text{ si } \sigma\x < 2
  \end{cases}
\end{equation*}
mientras que para la caracterización tenemos:
\begin{equation*}
  \begin{aligned}
    F^1 \cfbot &= \sigma \mapsto \begin{cases}
      \sigma &\text{ si } 2 \leq \sigma\x \\ 
      [\sigma\ |\ \x : 2] &\text{ si } 3-1 \leq \sigma\x < 2 \land \sigma\y = 0 \\ 
      \bot &\text{ si } \sigma\x < 3-1 \land \sigma\y \\ 
      \bot &\text{ si } \sigma\x < 2 \land \sigma\y \neq 0 
    \end{cases} \\ 
               &= \sigma \mapsto \begin{cases}
                 \sigma &\text{ si } 2 \leq \sigma\x \\ 
                 \bot &\text{ si } \sigma\x < 2 
               \end{cases}
  \end{aligned}
\end{equation*}
por lo que, entonces, se demuestra que se cumple para el caso base.

Ahora, si suponemos como HI que se cumple para $k \in \N_{\geq 1}$, veamos qué sucede con $k+1$:
\begin{equation*}
  \begin{aligned}
  F^{k+1} \cfbot &= \sigma \mapsto \begin{cases}
    F^k \cfbot [\sigma\ |\ \x : \sigma\x+1] &\text{ si } \sigma\x < 2 \land \sigma\y = 0 \\ 
    F^k \cfbot \sigma &\text{ si } \sigma\x < 2 \land \sigma\y \neq 0 \\ 
    \sigma &\text{ si } 2 \leq \sigma\x 
  \end{cases} \\ 
                 &= \sigma \mapsto \begin{cases}
                   [\sigma\ |\ \x : \sigma\x+1] &\text{ si } \sigma\x < 2 \land \sigma\y = 0 \land 2 \leq \sigma\x+1 \\ 
                   [\sigma\ |\ \x : \sigma\x+1\ |\ \x : 2] &\text{ si }\sigma\x < 2 \land \sigma\y = 0 \land 3-k \leq \sigma\x+1 < 2 \land \sigma\y = 0 \\ 
                   \bot &\text{ si }\sigma\x < 2 \land \sigma\y = 0 \land \sigma\x+1 < 3-k \land \sigma\y = 0 \\ 
                   \bot &\text{ si }\sigma\x < 2 \land \sigma\y = 0 \land \sigma\x+1 < 2 \land \sigma\y \neq 0 \\ 
                   \sigma &\text{ si }\sigma\x < 2 \land \sigma\y \neq 0 \land 2 \leq \sigma\x \\ 
                   [\sigma\ |\ \x : 2] &\text{ si } \sigma\x < 2 \land \sigma\y \neq 0 \land 3-k \leq \sigma\x < 2 \land \sigma\y = 0 \\ 
                   \bot &\text{ si }\sigma\x < 2 \land \sigma\y \neq 0 \land \sigma\x < 3-k \land \sigma\y = 0 \\ 
                   \bot &\text{ si }\sigma\x < 2 \land \sigma\y \neq 0 \land \sigma\x < 2 \land \sigma\y \neq 0 \\ 
                   \sigma &\text{ si } 2 \leq \sigma\x
                 \end{cases} \\ 
                 &= \sigma \mapsto \begin{cases}
                   [\sigma\ |\ \x : 2] &\text{ si } \sigma\x = 1 \land \sigma\y = 0 \\ 
                   [\sigma\ |\ \x : 2] &\text{ si } 3-(k+1) \leq \sigma\x < 1 \land \sigma\y = 0 \\ 
                   \bot &\text{ si } \sigma\x < 3-(k+1) \land \sigma\y = 0 \\ 
                   \bot &\text{ si } \sigma\x < 2 \land \sigma\y \neq 0 \\ 
                   \sigma &\text{ si } 2 \leq \sigma\x
                 \end{cases} \\ 
                 &= \sigma \mapsto \begin{cases}
                   \sigma &\text{ si } 2 \leq \sigma\x \\ 
                   [\sigma\ |\ \x : 2] &\text{ si } 3-(k+1) \leq \sigma\x < 2 \land \sigma\y = 0 \\ 
                   \bot &\text{ si } \sigma\x < 3-(k+1) \land \sigma\y = 0 \\ 
                   \bot &\text{ si } \sigma\x < 2 \land \sigma\y \neq 0
                 \end{cases}
  \end{aligned}
\end{equation*}
por lo que se demuestra el paso inductivo y, con ello, la caracterización propuesta. $\qed$

Sabiendo que esta caracterización es correcta, es sencillo ahora notar que por TMPF la semántica del while está dada por:
\begin{equation*}
  \supr_{i \in \N} F^i \cfbot = \supr_{i = 1}^\infty F^i \cfbot = \sigma \mapsto \begin{cases}
    \sigma &\text{ si } 2 \leq \sigma\x \\ 
    [\sigma\ |\ \x : 2] &\text{ si } \sigma\x < 2 \land \sigma\y = 0 \\ 
    \bot &\text{ si } \sigma\x < 2 \land \sigma\y \neq 0
  \end{cases}
\end{equation*}

\section*{Ejercicio 6}
Para este ejercicio, vamos a suponer que $\sem{\cwhile{b}{c}}\sigma \neq \bot$.

\subsection*{Item A}
Queremos demostrar que $\exists n \geq 0 : F^n \cfbot \sigma \neq \bot$.
Para ello, lo vamos a demostrar por el absurdo.
Digamos que no se cumple esta propiedad, es decir, $\forall n \geq 0 : F^n \cfbot \sigma = \bot$.
Como $F$ es continua al ser la semántica de un while, por el TMPF tenemos que $\sem{\cwhile{b}{c}} = \supr_{i \in \N} F^i \cfbot$.
Teniendo esto en cuenta, veamos que:
\begin{equation*}
  \begin{aligned}
    \sem{\cwhile{b}{c}} \sigma &= \left(\supr_{i \in \N} F^i \cfbot\right) \sigma \\ 
                               &= \supr_{i \in \N} (F^i \cfbot \sigma) \\ 
                               &= \supr_{i \in \N} \bot \\ 
                               &= \bot 
  \end{aligned}
\end{equation*}
por lo que se llega a un absurdo que vino de suponer que no se cumple la propiedad.

Luego, finalmente, se demuestra que $\exists n \geq 0 : F^n \cfbot \sigma \neq \bot$. $\qed$

\subsection*{Item B}
Queremos demostrar que $\sigma' = \sem{\cwhile{b}{c}}\sigma \Rightarrow \neg\sem{b}\sigma'$.
Para ello, primero supongamos que vale $\sigma' = \sem{\cwhile{b}{c}}\sigma$.
Por TMPF tenemos que $\sigma' = \sem{\cwhile{b}{c}}\sigma = \left(\supr_{i \in \N} F^i \cfbot\right)\sigma$.
Ahora, esto significa también que:
\begin{equation*}
  \begin{aligned}
    \sem{\cwhile{b}{c}} \sigma' &= \left(\supr_{i \in \N} F^i \cfbot\right) \sigma' \\ 
                                &= \supr_{i \in \N} (F^i \cfbot \sigma') \\ 
                                &= \supr_{i \in \N} \left(F^i \cfbot \left(\supr_{j \in \N} F^j \cfbot\right) \sigma\right) \\ 
                                &= \supr_{i \in \N} \left(F^i \cfbot \left(\supr_{j \in \N} F^j \cfbot \sigma\right)\right) \\ 
                                &= \supr_{i \in \N} \left(\supr_{j \in \N} F^{i + j} \cfbot \sigma\right) \\ 
                                &= \supr_{k \in \N} F^k \cfbot \sigma \\ 
                                &= \sigma'
  \end{aligned}
\end{equation*}

Ahora, tenemos que $\sigma' = \sem{\cwhile{b}{c}} \sigma' = \supr_{i \in \N} F^i \cfbot \sigma'$ y que:
\begin{equation*}
  F \sem{\cwhile{b}{c}} \sigma' = \begin{cases}
    \sem{\cwhile{b}{c}}_{\bbot} (\sem{c}\sigma') &\text{ si } \sem{b}\sigma' \\ 
    \sigma' &\text{ si } \neg\sem{b}\sigma'
  \end{cases}
\end{equation*}

Como $F \sem{\cwhile{b}{c}} \sigma' = F \left(\supr_{i \in \N} F^i \cfbot\right) \sigma' = \left(\supr_{i \in \N_{\geq 1}} F^i \cfbot\right) \sigma' = \left(\supr_{i \in \N} F^i \cfbot\right) \sigma' = \sigma'$, resulta que por la definición anterior para $F$ entonces se cumple $\neg\sem{b}\sigma'$.

Luego, con ello, se demuestra la implicación propuesta. $\qed$

\section*{Ejercicio 7}
Queremos demostrar o refutar las siguientes equivalencias usando semántica denotacional.
Veamos cada una de estas por separado.

\subsection*{Item A}
Se pretender analizar la equivalencia $\cwhile{\cfalse}{c} \equiv \cskip$.
Claramente estas son equivalentes por lo que nos vamos a concentrar en demostrar este hecho.

Veamos primero que, sea $w = \sem{\cwhile{\cfalse}{c}}$, entonces:
\begin{equation*}
  \begin{aligned}
    w \sigma &= \begin{cases}
      w_{\bbot} (\sem{c}\sigma) &\text{ si } \sem{\cfalse}\sigma \\ 
      \sigma &\text{ si }\neg\sem{\cfalse}\sigma 
    \end{cases} \\ 
             &= \sigma
  \end{aligned}
\end{equation*}

Mientras que del otro lado tenemos que:
\begin{equation*}
  \sem{\cskip} \sigma = \sigma
\end{equation*}

Motivo de ello, como $\forall \sigma \in \Sigma$ se cumple que $\sem{\cwhile{\cfalse}{c}}\sigma = \sigma = \sem{\cskip}\sigma$, se demuestra la equivalencia propuesta. $\qed$

\subsection*{Item B}
Se pretende analizar la equivalencia $\cwhile{b}{c} \equiv \cwhile{b}{(c;c)}$.
Claramente estas no son equivalentes y lo vamos a mostrar con un contraejemplo.
Sea $\cwhile{\x = 0}{\x := \x + 1} \equiv \cwhile{\x = 0}{(\x := \x + 1; \x := \x + 1)}$ la equivalencia para refutarla, $w_1$ y $w_2$ las semánticas de los dos whiles respectivamente, notemos que:
\begin{equation*}
  \begin{aligned}
    w_1 \sigma &= \begin{cases}
    {w_1}_{\bbot} (\sem{\x := \x + 1}\sigma) &\text{ si }\sem{\x = 0}\sigma \\ 
      \sigma &\text{ si }\neg\sem{\x = 0}\sigma 
    \end{cases} \\ 
             &= \begin{cases}
               w_1 [\sigma\ |\ \x : \sigma\x+1] &\text{ si } \sigma\x = 0 \\ 
               \sigma &\text{ si } \sigma\x \neq 0
             \end{cases} \\ 
             &= \begin{cases}
               w_1 [\sigma\ |\ \x : 1] &\text{ si }\sigma\x = 0 \\ 
               \sigma &\text{ si } \sigma\x \neq 0 
             \end{cases}
             \\ 
    w_2 \sigma &= \begin{cases}
      {w_2}_{\bbot} (\sem{\x := \x+1; \x := \x+1}\sigma) &\text{ si }\sem{\x = 0}\sigma \\ 
      \sigma &\text{ si }\neg\sem{\x = 0}\sigma 
    \end{cases} \\ 
               &= \begin{cases}
                 w_2 [\sigma\ |\ \x : \x+2] &\text{ si } \sigma\x = 0 \\ 
                 \sigma &\text{ si } \sigma\x \neq 0 
               \end{cases} \\ 
               &= \begin{cases}
                 w_2 [\sigma\ |\ \x : 2] &\text{ si }\sigma\x = 0 \\ 
                 \sigma &\text{ si } \sigma\x \neq 0
               \end{cases}
  \end{aligned}
\end{equation*}

Con esto en mente, entonces supongamos $F_1, F_2 \in (\cdom) \to (\cdom)$ tales que:
\begin{equation*}
  \begin{aligned}
    F w_1 \sigma &= \begin{cases}
      w_1 [\sigma\ |\ \x : 1] &\text{ si } \sigma\x = 0 \\ 
      \sigma &\text{ si }\sigma\x \neq 0
    \end{cases} \\ 
    \\ 
      F w_2 \sigma &= \begin{cases}
        w_2 [\sigma\ |\ \x : 2] &\text{ si }\sigma\x = 0 \\ 
        \sigma &\text{ si }\sigma\x \neq 0
      \end{cases}
  \end{aligned}
\end{equation*}

Ahora, al ser estas semánticas del while, veamos algunos $F^i \cfbot$ para calcularlas gracias al TMPF:
\begin{equation*}
  \begin{aligned}
    F_1^0 \cfbot &= \cfbot \\ 
    \\ 
    F_1^1 \cfbot &= \sigma \mapsto \begin{cases}
      \cfbot [\sigma\ |\ \x : 1] &\text{ si }\sigma\x = 0 \\ 
      \sigma &\text{ si }\sigma\x \neq 0
    \end{cases} \\ 
                 &= \sigma \mapsto \begin{cases}
                   \bot &\text{ si }\sigma\x = 0 \\ 
                   \sigma &\text{ si }\sigma\x \neq 0
                 \end{cases}
    \\ 
      F_1^2 \cfbot &= \sigma \mapsto \begin{cases}
        F_1 [\sigma\ |\ \x : 1] &\text{ si }\sigma\x = 0 \\ 
        \sigma &\text{ si }\sigma\x \neq 0 
      \end{cases} \\ 
                   &= \sigma \mapsto \begin{cases}
                     \bot &\text{ si }\sigma\x = 0 \land 1 = 0 \\ 
                     [\sigma\ |\ \x : 1] &\text{ si }\sigma\x = 0 \land 1 \neq 0 \\ 
                     \sigma &\text{ si }\sigma\x \neq 0
                   \end{cases} \\ 
                   &= \sigma \mapsto \begin{cases}
                     [\sigma\ |\ \x : 1] &\text{ si } \sigma\x = 0 \\ 
                     \sigma &\text{ si }\sigma\x \neq 0
                   \end{cases}
  \end{aligned}
\end{equation*}
y, de forma totalmente análoga, podemos llegar a que:
\begin{equation*}
  \begin{aligned}
    F_2^2 \cfbot &= \sigma \mapsto \begin{cases}
      [\sigma\ |\ \x : 2] &\text{ si }\sigma\x = 0 \\ 
      \sigma &\text{ si }\sigma\x \neq 0
    \end{cases}
  \end{aligned}
\end{equation*}

Ahora, dado que tenemos las cadenas $F_1^0 \cfbot \fleq F_1^1 \cfbot \fleq \dots$ y $F_2^0 \cfbot \fleq F_2^1 \cfbot \fleq \dots$ y tanto $F_1^2$ como $F_2^2$ no tienen a bottom en la imagen, entonces ambos son el supremo de sus cadenas:
\begin{equation*}
  \begin{aligned}
    \supr_{i \in \N} F_1^i \cfbot &= \sigma \mapsto \begin{cases}
      [\sigma\ |\ \x : 1] &\text{ si }\sigma\x = 0 \\ 
      \sigma &\text{ si }\sigma\x \neq 0 
    \end{cases} \\ 
    \\ 
    \supr_{i \in \N} F_2^i \cfbot &= \sigma \mapsto \begin{cases}
      [\sigma\ |\ \x : 2] &\text{ si }\sigma\x = 0 \\ 
      \sigma &\text{ si }\sigma\x \neq 0 
    \end{cases} \\ 
  \end{aligned}
\end{equation*}

Finalmente, entonces, esto muestra que sus semánticas son distintas por lo que no se cumple la equivalencia al mostrar un contraejemplo de ello.

\subsection*{Item C}
Queremos analizar la equivalencia dada por $(\cwhile{b}{c}); \cif{b}{c_0}{c_1} \equiv (\cwhile{b}{c}); c_1$.
Claramente son equivalentes por lo que nos concentraremos en demostrarlo.
Para ello, sea $w = \sem{\cwhile{b}{c}}$, las semánticas son
\begin{equation*}
  \begin{aligned}
    \sem{(\cwhile{b}{c});\cif{b}{c_0}{c_1}} \sigma &= \sem{\cif{b}{c_0}{c_1}}_{\bbot} (w \sigma) \\ 
                                                   &= \begin{cases}
                                                     \bot &\text{ si } w\sigma = \bot \\ 
                                                     \sem{c_0} (w \sigma) &\text{ si } w\sigma \neq \bot \land \sem{b}(w\sigma) \\ 
                                                     \sem{c_1} (w \sigma) &\text{ si }w\sigma \neq \bot \land \neg\sem{b}(w\sigma)
                                                   \end{cases} \\ 
                                                   \\ 
    \sem{(\cwhile{b}{c});c_1} \sigma &= \sem{c_1}_{\bbot} (w\sigma) \\ 
                                     &= \begin{cases}
                                       \bot &\text{ si }w\sigma = \bot \\ 
                                       \sem{c_1}(w\sigma) &\text{ si }w\sigma \neq \bot
                                     \end{cases}
  \end{aligned}
\end{equation*}

Por el ejercicio $(6b)$ sabemos que $\sigma' = w\sigma \Rightarrow \neg\sem{b}\sigma'$.
Por ello, entonces en la semántica del primer comando no se considera el 2do caso y, finalmente, las semánticas de ambos comandos son iguales.
Con ello, se demuestra la equivalencia para $\sigma \in \Sigma$. $\qed$

\section*{Ejercicio 8}
En este ejercicio se pretende analizar las siguientes propuestas de definiciones como syntactic sugar de $\cfor{v}{e_0}{e_1}{c}$.
Estas son:
\begin{itemize}
  \item $v := e_0; \cwhile{v \leq e_1}{c}; v := v + 1$
  \item $\cnewvar{v}{e_0}{\cwhile{v \leq e_1}{c}; v := v + 1}$
  \item $\cnewvar{w}{e_1}{\cnewvar{v}{e_0}{\cwhile{v \leq w}{c; v := v + 1}}}$
\end{itemize}

Dado que se considera en este for un rango específico $[e_0, e_1]$, donde ambas son expresiones enteras, entonces la mejor opción sería la 3er definición debido a que antes de iniciar con la iteración, fija los valores de los extremos donde esta se va a realizar (mediante el uso de variables locales).

\section*{Siguientes ejercicios}
No se continúa con los siguientes ejercicios de la presente guía.
Si poseo más tiempo para realizarlos, los haré.
Independientemente de ello, algunas menciones importantes son:
\begin{itemize}
  \item \textit{Ejercicio 9}: El caso \textbf{while} del TC se encuentra demostrado en archivos del aula virtual.
                 Por ejemplo, en \href{https://famaf-consultas.aulavirtual.unc.edu.ar/pluginfile.php/123179/mod_folder/content/0/TCwhile.pdf?forcedownload=1}{este} del aula virtual 2021 en FAMAF consultas.
  \item \textit{Ejercicio 10}: La solución a este ejercicio puede encontrarse en un \href{https://famaf-consultas.aulavirtual.unc.edu.ar/pluginfile.php/61374/mod_resource/content/2/P4E10.pdf}{archivo} en el aula virtual 2020 en FAMAF consultas.
\end{itemize}

\end{document}
