\documentclass{article}
\usepackage[spanish]{babel}
\usepackage{geometry}
\usepackage{titling}
\usepackage{hyperref}
\usepackage{amsmath}
\usepackage{amsthm}
\usepackage{amssymb}
\usepackage{graphicx}
\usepackage{caption}
\usepackage{subcaption}
\usepackage{stmaryrd}
\usepackage{enumitem}
\usepackage[dvipsnames]{xcolor}

\geometry{
  a4paper,
  total = {170mm, 257mm},
  left = 20mm,
  top = 20mm,
}
\graphicspath{ {./images/} }

\newtheorem{theorem}{Teorema}[section]
\newtheorem{lemma}[theorem]{Lema}

\newtheorem*{theorem*}{Teorema}
\newtheorem*{lemma*}{Lema}

\newcommand{\addfig}[2]{\begin{figure}[!htb] \centering \includegraphics[width=#1\textwidth]{#2}\end{figure}}

\newcommand{\aexp}[1]{\langle\text{#1}\rangle}
\newcommand{\intexp}{\aexp{intexp}}
\newcommand{\var}{\aexp{var}}
\newcommand{\assert}{\aexp{assert}}
\newcommand{\boolexp}{\aexp{boolexp}}
\newcommand{\comm}{\aexp{comm}}
\newcommand{\sem}[1]{\left\llbracket #1\right\rrbracket}

\newcommand{\N}{\mathbb{N}}
\newcommand{\Z}{\mathbb{Z}}
\newcommand{\B}{\mathbb{B}}

\newcommand{\x}{\textbf{x}}
\newcommand{\y}{\textbf{y}}
\newcommand{\z}{\textbf{z}}

\newcommand{\supr}{\bigsqcup\limits}

\newcommand{\concat}{\texttt{ ++ }}

\newcommand{\fleq}{\sqsubseteq}
\newcommand{\cdom}{\Sigma \to \Sigma_\bot}
\newcommand{\cdomf}{\Sigma \to \Sigma_\bot'}
\newcommand{\cdomfo}{\Sigma \to \Omega}
\newcommand{\cfbot}{\bot_{\cdom}}
\newcommand{\cfbotf}{\bot_{\cdomf}}
\newcommand{\cfbotfo}{\bot_{\cdomfo}}
\newcommand{\bbot}{\bot\!\!\!\bot}
\newcommand{\ctrue}{\textbf{true}}
\newcommand{\cfalse}{\textbf{false}}
\newcommand{\cskip}{\textbf{skip}}
\newcommand{\cif}[3]{\textbf{if }#1\textbf{ then }#2\textbf{ else }#3}
\newcommand{\cnewvar}[3]{\textbf{newvar }#1 := #2\textbf{ in }#3}
\newcommand{\cwhile}[2]{\textbf{while }#1\textbf{ do }#2}
\newcommand{\cfor}[4]{\textbf{for }#1 := #2\text{ to }#3\text{ do }#4}
\newcommand{\cfail}{\textbf{fail}}
\newcommand{\ccatch}[2]{\textbf{catchin }#1\textbf{ with }#2}
\newcommand{\cabort}[1]{\langle\textbf{abort}, #1\rangle}
\newcommand{\cout}[1]{\langle #1\rangle}

\newcommand{\iterm}[1]{\iota_\text{term}\left(#1\right)}
\newcommand{\iabort}[1]{\iota_\text{abort}\left(#1\right)}
\newcommand{\iout}[2]{\iota_\text{out}\left(#1,\ #2\right)}
\newcommand{\iin}[2]{\iota_\text{in}\left(\lambda #1 \in \Z .\ #2\right)}

\newcommand{\ibot}{\iota_\bot}

\title{Trabajo práctico N° 8}
\author{Emanuel Nicolás Herrador}
\date{Mayo 2025}

\makeatletter
\def\@maketitle{%
  \newpage
  \null
  \vskip 1em%
  \begin{center}%
  \let \footnote \thanks
    {\LARGE \@title \par}%
    \vskip 1em%
    {\large \@date}%
  \end{center}%
  \par
  \vskip 1em}
\makeatother

\begin{document}

\maketitle

\noindent\begin{tabular}{@{}ll}
	Estudiante & \theauthor \\
\end{tabular}

\section*{Ejercicio 1}
Veamos la semántica denotacional en $D_\infty$ para cada uno de los términos separados en items.

\subsection*{Item A}
Queremos ver la semántica de $M = \lambda f.\lambda x.\ f(fx)$.
Veamos que:
\begin{equation*}
  \begin{aligned}
    \sem{M}\eta &= \sem{\lambda f.\lambda x.\ f(fx)}\eta \\ 
                &= \psi(\lambda c.\ \sem{\lambda x.\ f(fx)}[\eta\ |\ f : c]) \\ 
                &= \psi(\lambda c.\ \psi(\lambda d.\ \sem{f(fx)}[\eta\ |\ f : c\ |\ x : d])) \\ 
                &= \psi(\lambda c.\ \psi(\lambda d.\ \varphi(\sem{f}[\eta\ |\ f : c\ |\ x : d])(\sem{fx}[\eta\ |\ f : c\ |\ x : d]))) \\ 
                &= \psi(\lambda c.\ \psi(\lambda d.\ \varphi c (\sem{fx}[\eta\ |\ f : c\ |\ x : d]))) \\ 
                &= \psi(\lambda c.\ \psi(\lambda d.\ \varphi c (\varphi(\sem{f}[\eta\ |\ f : c\ |\ x : d])(\sem{x}[\eta\ |\ f : c\ |\ x : d])))) \\ 
                &= \psi(\lambda c.\ \psi(\lambda d.\ \varphi c (\varphi c d))) \\ 
  \end{aligned}
\end{equation*}

\subsection*{Item B}
Se pretende ver la semántica de $N = \lambda z.\lambda y.\ z$.
Veamos que:
\begin{equation*}
  \begin{aligned}
    \sem{N}\eta &= \sem{\lambda z.\lambda y.\ z}\eta \\ 
                &= \psi(\lambda c.\ \sem{\lambda y.\ z}[\eta\ |\ z : c]) \\ 
                &= \psi(\lambda c.\ \psi(\lambda d.\ \sem{z}[\eta\ |\ z : c\ |\ y : d])) \\ 
                &= \psi(\lambda c.\ \psi(\lambda d.\ c)) \\ 
  \end{aligned}
\end{equation*}

\subsection*{Item C}
Ahora queremos ver la semántica de $MN$.
Para ello, veamos que:
\begin{equation*}
  \begin{aligned}
    \sem{MN}\eta &= \varphi(\sem{M}\eta)(\sem{N}\eta) \\ 
                 &= \varphi\ (\psi(\lambda c.\ \psi(\lambda d.\ \varphi c (\varphi c d))))\ (\psi(\lambda e.\ \psi(\lambda f.\ e))) \\ 
                 &= (\lambda c.\ \psi(\lambda d.\ \varphi c (\varphi c d))) (\psi(\lambda e.\ \psi(\lambda f.\ e))) \\ 
                 &= \psi(\lambda d.\ \varphi\ (\psi(\lambda e.\ \psi(\lambda f.\ e)))\ (\varphi\ (\psi(\lambda e.\ \psi(\lambda f.\ e)))\ d)) \\ 
                 &= \psi(\lambda d.\ (\lambda e.\ \psi(\lambda f.\ e))\ ((\lambda e.\ \psi(\lambda f.\ e))\ d)) \\ 
                 &= \psi(\lambda d.\ (\lambda e.\ \psi(\lambda f.\ e))\ (\psi(\lambda f.\ d))) \\ 
                 &= \psi(\lambda d.\ \psi(\lambda f.\ \psi(\lambda f'.\ d)))
  \end{aligned}
\end{equation*}

\section*{Ejercicio 2}
Los enuncio pero no los voy a demostrar.

\begin{theorem*}[Renombre]
  Si $v' \notin FVe - \{v\}$, entonces $\sem{\lambda v'.\ (e / v \to v')} = \sem{\lambda v.\ e}$.
\end{theorem*}

\begin{theorem*}[Coincidencia]
  Si $\eta w = \eta' w$ para todo $w \in FVe$, entonces $\sem{e}\eta = \sem{e}\eta'$.
\end{theorem*}

\begin{theorem*}[Corrección de la regla $\beta$]
  $\sem{(\lambda v.\ e)e'} = \sem{e / v \to e'}$.
\end{theorem*}

\begin{theorem*}[Corrección de la regla $\eta$]
  Si $v \notin FVe$, entonces $\sem{\lambda v.\ ev} = \sem{e}$.
\end{theorem*}

\section*{Ejercicio 3}
El objetivo es dar un término cerrado $M$ cuya denotación en la semántica normal sea:
\begin{enumerate}[label=(\alph*)]
  \item Distinto a $\bot$ pero que para todos $N$ y $\eta$, $\sem{MN}\eta = \bot$:

    Podemos tomar $M = \lambda v.\ (\Delta\Delta)$ porque $\sem{M}\eta \neq \bot$ y:
    \begin{equation*}
      \begin{aligned}
        \sem{MN}\eta &= \varphi_{\bbot}\ (\sem{M}\eta)\ (\sem{N}\eta) \\ 
                     &= \varphi_{\bbot}\ (\sem{\lambda v.\ (\Delta\Delta)}\eta)\ (\sem{N}\eta) \\ 
                     &= \varphi_{\bbot}\ (\ibot\psi(\lambda d.\ \sem{\Delta\Delta}[\eta\ |\ v : d]))\ (\sem{N}\eta) \\ 
                     &= \varphi_{\bbot}\ (\ibot\psi(\lambda d.\ \bot))\ (\sem{N}\eta) \\ 
                     &= (\lambda d.\ \bot)\ (\sem{N}\eta) \\ 
                     &= \bot
      \end{aligned}
    \end{equation*}

  \item Distinto a $\bot$ y $\sem{M(\Delta\Delta)}\eta \neq \bot$:

    Podemos tomar $M = \lambda x.\lambda y.\ y$ porque $\sem{M}\eta \neq \bot$ y:
    \begin{equation*}
      \begin{aligned}
        \sem{M(\Delta\Delta)}\eta &= \varphi_{\bbot}\ (\sem{M}\eta)\ (\sem{\Delta\Delta}\eta) \\ 
                                  &= \varphi_{\bbot}\ (\sem{\lambda x.\lambda y.\ y}\eta)\ \bot \\ 
                                  &= \varphi_{\bbot}\ (\ibot\psi(\lambda d.\ \sem{\lambda y.\ y}[\eta\ |\ x : d]))\ \bot \\ 
                                  &= \varphi_{\bbot}\ (\ibot\psi(\lambda d.\ \ibot\psi(\lambda e.\ \sem{y}[\eta\ |\ x : d\ |\ y : e])))\ \bot \\ 
                                  &= \varphi_{\bbot}\ (\ibot\psi(\lambda d.\ \ibot\psi(\lambda e.\ e)))\ \bot \\ 
                                  &= (\lambda d.\ \ibot\psi(\lambda e.\ e))\ \bot \\ 
                                  &= \ibot\psi(\lambda e.\ e) \\ 
                                  &\neq \bot
      \end{aligned}
    \end{equation*}
\end{enumerate}

\section*{Ejercicio 4}
La semántica eager de $\sem{M(\Delta\Delta)}\eta$ dado en $(3b)$ es $\bot$ porque en el orden eager siempre se evalúan también los argumentos.
Motivo de ello, como $\sem{\Delta\Delta}\eta = \bot$, la semántica de la aplicación también es $\bot$.

Se puede ver por la definición porque en eager la aplicación se define como $\sem{e_0e_1}\eta = \varphi_{\bbot}\ (\sem{e_0}\eta)_{\bbot}\ (\sem{e_1}\eta)$.
Luego, si $\sem{e_1}\eta = \bot$, por el segundo $_{\bbot}$ todo es $\bot$.

\section*{Ejercicio 5}
Si consideramos la semántica denotacional normal del cálculo lambda, entonces tenemos que:
\begin{enumerate}[label=(\alph*)]
  \item \textit{Teorema de sustitución}: Vale.
  \item \textit{Corrección de la regla $\beta$}: Vale.
  \item \textit{Corrección de la regla $\eta$}: No vale por el caso $\sem{\lambda v.\ (\Delta\Delta)v}\eta \neq \bot = \sem{\Delta\Delta}\eta$.
\end{enumerate}

\section*{Ejercicio 6}
Si consideramos la semántica denotacional eager del cálculo lambda, entonces tenemos que:
\begin{enumerate}[label=(\alph*)]
  \item \textit{Teorema de sustitución}: No vale porque $\sem{\delta w}\eta \in V_\bot$ y $\eta' w \in V$ por lo que son incomparables.
  \item \textit{Corrección de la regla $\beta$}: No vale por el caso $\sem{(\lambda v.\lambda x.\ x)(\Delta\Delta)}\eta = \bot \neq \sem{\lambda x.\ x}$.
  \item \textit{Corrección de la regla $\eta$}: No vale por mismo caso que con orden normal.
\end{enumerate}

\section*{Ejercicio 7}
\begin{theorem*}[TS para Eager]
  Si $\sem{\delta w}\eta = \ibot(\eta'w)$ para toda $w \in FVe$, entonces $\sem{e/d}\eta = \sem{e}\eta'$.
\end{theorem*}

\section*{Ejercicio 8}
Se pretende ver cuáles de las siguientes afirmaciones son verdaderas y cuáles falsas, justificando el porqué.

En el caso de los items $(a)$ a $(d)$, estos son imprecisos porque los entornos son distintos.

Respecto al item $(e)$, no se define un isomorfismo en normal con $\varphi_{\bbot}$ y $\ibot\psi$ porque $\forall f \in [D \to D],\ \ibot\psi f \neq \bot_D$ por lo que la segunda no es suryectiva.

Por último, respecto al item $(f)$, este es falso porque:
\begin{equation*}
  \begin{aligned}
    \varphi_{\bbot} &\in V_\bot \to [V \to V_\bot] \\ 
    \ibot \circ \psi &\in [V \to V_\bot] \to V_\bot \\ 
    \varphi_{\bbot} \circ (\ibot \circ \psi) &\in V_\bot \to V_\bot 
  \end{aligned}
\end{equation*}

Luego, los dominios son distintos por lo que no se cumple.

\end{document}
