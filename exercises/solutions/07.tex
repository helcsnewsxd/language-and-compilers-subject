\documentclass{article}
\usepackage[spanish]{babel}
\usepackage{geometry}
\usepackage{titling}
\usepackage{hyperref}
\usepackage{amsmath}
\usepackage{amsthm}
\usepackage{amssymb}
\usepackage{graphicx}
\usepackage{caption}
\usepackage{subcaption}
\usepackage{stmaryrd}
\usepackage{enumitem}
\usepackage[dvipsnames]{xcolor}

\geometry{
  a4paper,
  total = {170mm, 257mm},
  left = 20mm,
  top = 20mm,
}
\graphicspath{ {./images/} }

\newtheorem{theorem}{Teorema}[section]
\newtheorem{lemma}[theorem]{Lema}

\newtheorem*{theorem*}{Teorema}
\newtheorem*{lemma*}{Lema}

\newcommand{\addfig}[2]{\begin{figure}[!htb] \centering \includegraphics[width=#1\textwidth]{#2}\end{figure}}

\newcommand{\aexp}[1]{\langle\text{#1}\rangle}
\newcommand{\intexp}{\aexp{intexp}}
\newcommand{\var}{\aexp{var}}
\newcommand{\assert}{\aexp{assert}}
\newcommand{\boolexp}{\aexp{boolexp}}
\newcommand{\comm}{\aexp{comm}}
\newcommand{\sem}[1]{\left\llbracket #1\right\rrbracket}

\newcommand{\N}{\mathbb{N}}
\newcommand{\Z}{\mathbb{Z}}
\newcommand{\B}{\mathbb{B}}

\newcommand{\x}{\textbf{x}}
\newcommand{\y}{\textbf{y}}
\newcommand{\z}{\textbf{z}}

\newcommand{\supr}{\bigsqcup\limits}

\newcommand{\concat}{\texttt{ ++ }}

\newcommand{\fleq}{\sqsubseteq}
\newcommand{\cdom}{\Sigma \to \Sigma_\bot}
\newcommand{\cdomf}{\Sigma \to \Sigma_\bot'}
\newcommand{\cdomfo}{\Sigma \to \Omega}
\newcommand{\cfbot}{\bot_{\cdom}}
\newcommand{\cfbotf}{\bot_{\cdomf}}
\newcommand{\cfbotfo}{\bot_{\cdomfo}}
\newcommand{\bbot}{\bot\!\!\!\bot}
\newcommand{\ctrue}{\textbf{true}}
\newcommand{\cfalse}{\textbf{false}}
\newcommand{\cskip}{\textbf{skip}}
\newcommand{\cif}[3]{\textbf{if }#1\textbf{ then }#2\textbf{ else }#3}
\newcommand{\cnewvar}[3]{\textbf{newvar }#1 := #2\textbf{ in }#3}
\newcommand{\cwhile}[2]{\textbf{while }#1\textbf{ do }#2}
\newcommand{\cfor}[4]{\textbf{for }#1 := #2\text{ to }#3\text{ do }#4}
\newcommand{\cfail}{\textbf{fail}}
\newcommand{\ccatch}[2]{\textbf{catchin }#1\textbf{ with }#2}
\newcommand{\cabort}[1]{\langle\textbf{abort}, #1\rangle}
\newcommand{\cout}[1]{\langle #1\rangle}

\newcommand{\iterm}[1]{\iota_\text{term}\left(#1\right)}
\newcommand{\iabort}[1]{\iota_\text{abort}\left(#1\right)}
\newcommand{\iout}[2]{\iota_\text{out}\left(#1,\ #2\right)}
\newcommand{\iin}[2]{\iota_\text{in}\left(\lambda #1 \in \Z .\ #2\right)}

\title{Trabajo práctico N° 7}
\author{Emanuel Nicolás Herrador}
\date{Mayo 2025}

\makeatletter
\def\@maketitle{%
  \newpage
  \null
  \vskip 1em%
  \begin{center}%
  \let \footnote \thanks
    {\LARGE \@title \par}%
    \vskip 1em%
    {\large \@date}%
  \end{center}%
  \par
  \vskip 1em}
\makeatother

\begin{document}

\maketitle

\noindent\begin{tabular}{@{}ll}
	Estudiante & \theauthor \\
\end{tabular}

\section*{Ejercicio 1}
Veamos cada una de las expresiones en un item separado para mayor claridad.

\subsection*{Item A}
\begin{equation*}
  \begin{aligned}
    (\lambda f. \lambda x. f(fx)) (\lambda z. \lambda x. \lambda y. zyx) (\lambda z. \lambda w. z) &\equiv (\lambda f. \lambda t. f(ft))(\lambda z. \lambda x. \lambda y. zyx)(\lambda z.\lambda w. z) \\ 
                                                                                                   &\to (\lambda t. (\lambda z.\lambda x.\lambda y. zyx)((\lambda z.\lambda x.\lambda y. zyx)t))(\lambda z.\lambda w. z) \\ 
                                                                                                   &\to (\lambda t. (\lambda z.\lambda x.\lambda y. zyx)(\lambda x.\lambda y. tyx))(\lambda z.\lambda w. z) \\ 
                                                                                                   &\equiv (\lambda t. (\lambda z.\lambda s.\lambda u. zus)(\lambda x.\lambda y. tyx))(\lambda z.\lambda w. z) \\ 
                                                                                                   &\to (\lambda t.\lambda s.\lambda u. (\lambda x.\lambda y. tyx)us)(\lambda z.\lambda w. z) \\ 
                                                                                                   &\to (\lambda t.\lambda s.\lambda u. (\lambda y. tyu)s)(\lambda z.\lambda w. z) \\ 
                                                                                                   &\to (\lambda t.\lambda s.\lambda u. tsu)(\lambda z.\lambda w. z) \\ 
                                                                                                   &\to \lambda s.\lambda u. (\lambda z.\lambda w. z)su \qquad\text{forma canónica}\\
                                                                                                   &\to \lambda s.\lambda u. (\lambda w. s)u \\ 
                                                                                                   &\to \lambda s.\lambda u. s
  \end{aligned}
\end{equation*}

Luego, $(\lambda f. \lambda x. f(fx)) (\lambda z. \lambda x. \lambda y. zyx) (\lambda z. \lambda w. z) \to^* \lambda s.\lambda u. (\lambda z.\lambda w. z)su \to^* \lambda s.\lambda u. s$.

\subsection*{Item B}
\begin{equation*}
  \begin{aligned}
    (\lambda z. zz)(\lambda f.\lambda x. f(fx)) &\to (\lambda f.\lambda x. f(fx))(\lambda f.\lambda x. f(fx)) \\ 
                                                &\equiv (\lambda f.\lambda t. f(ft))(\lambda f.\lambda x. f(fx)) \\ 
                                                &\to \lambda t. (\lambda f.\lambda x. f(fx))((\lambda f.\lambda x. f(fx))t) \qquad\text{forma canónica} \\ 
                                                &\to \lambda t. (\lambda f.\lambda x. f(fx))(\lambda x. t(tx)) \\ 
                                                &\equiv \lambda t. (\lambda f.\lambda y. f(fy))(\lambda x. t(tx)) \\ 
                                                &\to \lambda t.\lambda y. (\lambda x. t(tx))((\lambda x. t(tx))y) \\ 
                                                &\to \lambda t.\lambda y. (\lambda x. t(tx))(t(ty)) \\ 
                                                &\to \lambda t.\lambda y. t(t(t(ty))) 
  \end{aligned}
\end{equation*}

Luego, $(\lambda z. zz)(\lambda f.\lambda x. f(fx)) \to^* \lambda t. (\lambda f.\lambda x. f(fx))((\lambda f.\lambda x. f(fx))t) \to^* \lambda t.\lambda y. t(t(t(ty)))$.

\section*{Ejercicio 2}
Consideramos lo siguiente:
\begin{equation*}
  \begin{aligned}
    TRUE &= \lambda x.\lambda y. x \\ 
    FALSE &= \lambda x.\lambda y. y \\ 
    NOT &= \lambda b.\lambda x.\lambda y. byx \\ 
    AND &= \lambda b.\lambda c.\lambda x.\lambda y. b(cxy)y \\ 
    IF &= \lambda b.\lambda x.\lambda y. bxy 
  \end{aligned}
\end{equation*}

Y, en base a ello, tenemos que demostrar varias relaciones de reducción.
Veamos cada una de ella en items separados para mayor claridad.

\subsection*{Item A}
\begin{lemma*}
  $NOT\ TRUE \to^* FALSE$.
\end{lemma*}
\begin{proof}
  Para demostrarlo, veamos la reducción correspondiente:
  \begin{equation*}
    \begin{aligned}
      NOT\ TRUE &\equiv (\lambda b.\lambda x.\lambda y. byx)\ TRUE \\
               &\to \lambda x.\lambda y. TRUEyx \\ 
               &\equiv \lambda x.\lambda y. (\lambda s.\lambda t. s)yx \\ 
               &\to \lambda x.\lambda y. (\lambda t. y)x \\ 
               &\to \lambda x.\lambda y. y \\ 
               &\equiv FALSE
    \end{aligned}
  \end{equation*}
  por lo que llegamos a que $NOT\ TRUE \to^* FALSE$ y se demuestra.
\end{proof}

\subsection*{Item B}
\begin{lemma*}
  $IF\ TRUE\ e_0 e_1 \to^* e_0$.
\end{lemma*}
\begin{proof}
  Para demostrarlo, veamos la siguiente reducción:
  \begin{equation*}
    \begin{aligned}
      IF\ TRUE\ e_0 e_1 &\equiv (\lambda b. \lambda x. \lambda y. bxy)\ TRUE\ e_0 e_1 \\ 
                        &\to (\lambda x. \lambda y. TRUE\ xy) e_0 e_1 \\ 
                        &\to (\lambda y. TRUE\ e_0 y) e_1 \\ 
                        &\to TRUE\ e_0 e_1 \\ 
                        &\equiv (\lambda x.\lambda y. x) e_0 e_1 \\ 
                        &\to (\lambda y. e_0) e_1 \\ 
                        &\to e_0
    \end{aligned}
  \end{equation*}
  Por ello, entonces, $IF\ TRUE\ e_0 e_1 \to^* e_0$ y se demuestra.
\end{proof}

\subsection*{Item C}
\begin{lemma*}
  $AND\ TRUE\ TRUE \to^* TRUE$.
\end{lemma*}
\begin{proof}
  Para demostrarlo, veamos la siguiente reducción:
  \begin{equation*}
    \begin{aligned}
      AND\ TRUE\ TRUE &\equiv (\lambda b.\lambda c.\lambda x.\lambda y. b(cxy)y)\ TRUE\ TRUE \\ 
                      &\to (\lambda c.\lambda x.\lambda y.\ TRUE(cxy)y)\ TRUE \\ 
                      &\to \lambda x.\lambda y.\ TRUE(TRUE\ xy)y \\ 
                      &\equiv \lambda x.\lambda y. (\lambda s.\lambda t. s)(TRUE\ xy)y \\ 
                      &\to \lambda x.\lambda y.\ (\lambda t.\ (TRUE\ xy))y \\ 
                      &\to \lambda x.\lambda y.\ TRUE\ xy \\ 
                      &\equiv \lambda x.\lambda y.\ (\lambda s.\lambda t. s)xy \\ 
                      &\to \lambda x.\lambda y. (\lambda t. x)y \\ 
                      &\to \lambda x.\lambda y. x \\ 
                      &\equiv TRUE
    \end{aligned}
  \end{equation*}
  Por ello, $AND\ TRUE\ TRUE \to^* TRUE$ y se demuestra.
\end{proof}

\subsection*{Item D}
\begin{lemma*}
  $AND\ FALSE\ e \to^* FALSE$.
\end{lemma*}
\begin{proof}
  Para demostrarlo, veamos la siguiente reducción:
  \begin{equation*}
    \begin{aligned}
      AND\ FALSE\ e &\equiv (\lambda b.\lambda c.\lambda x.\lambda y.\ b(cxy)y)\ FALSE\ e \\ 
                    &\to (\lambda c.\lambda x.\lambda y.\ FALSE(cxy)y)e \\ 
                    &\to \lambda x.\lambda y.\ FALSE(exy)y \\ 
                    &\equiv \lambda x.\lambda y. (\lambda s.\lambda t. t)(exy)y \\ 
                    &\to \lambda x.\lambda y. (\lambda t. t)y \\ 
                    &\to \lambda x.\lambda y. y \\ 
                    &\equiv FALSE
    \end{aligned}
  \end{equation*}
  Por ello, tenemos que $AND\ FALSE\ e \to^* FALSE$ y se demuestra.
\end{proof}

\section*{Ejercicio 3}
Queremos ver cuáles de las siguientes afirmaciones son verdaderas y cuáles falsas justificando el porqué.
Veamos cada una por separado en una enumeración:
\begin{enumerate}[label=(\alph*)]
  \item ``Toda expresión lambda cerrada tiene forma normal''.

    \textbf{Falso} con $\lambda y.\ \Delta\Delta \equiv \lambda y. (\lambda x. xx)(\lambda x. xx)$ como contraejemplo dado que sabemos que no tiene forma normal y es una expresión lambda cerrada.

  \item ``Toda expresión lambda cerrada tiene forma canónica''.

    \textbf{Verdadero}. En particular, una expresión lambda es una abstracción, la cual es la que consideramos como forma canónica en el Cálculo Lambda.

  \item ``Toda forma canónica cerrada es forma normal''.

    \textbf{Falso} con $\lambda y. \Delta\Delta$ como contraejemplo.

  \item ``Toda forma normal cerrada es forma canónica''.

    \textbf{Verdadero}.
    \begin{proof}
      Sea $e$ una forma normal cerrada, por propiedad, sabemos que una aplicación cerrada no puede ser una forma normal.
      Por contrarrecíproca, esto implica que una forma normal no puede ser una aplicación cerrada.
      Luego, $e$ no es una aplicación cerrada por lo que, o bien es una variable (no porque contradice el hecho de que es cerrada), o bien es una abstracción.
      Por ello, llegamos a que $e \equiv \lambda v. e'$ para $v$ variable y $e'$ expresión.

      Finalmente, entonces, esto implica que $e$ es una forma canónica por lo que se demuestra.
    \end{proof}
\end{enumerate}

\section*{Ejercicio 4}
Se pretende demostrar el siguiente lema:
\begin{lemma*}
  Una aplicación cerrada no puede ser una forma normal.
\end{lemma*}

Para ello, veamos que:
\begin{proof}
  Sea $e$ una aplicación cerrada, entonces $e = e_0 e_1 \dots e_n$ para $e_i$ expresiones ($0 \leq i \leq n$) donde $e_0$ no es una aplicación ni una variable y $e \geq 1$.
  Luego, $e_0$ es una abstracción por lo que $e$ contiene el redex $e_0 e_1$.
  Por ello, no es forma normal y se demuestra el lema.
\end{proof}

\section*{Ejercicio 5}
Veamos cada expresión por separado.

\subsection*{Item A}
Tenemos la expresión $(\lambda f.\lambda x.\ f(fx))(\lambda z.\lambda x.\lambda y.\ zyx)(\lambda z.\lambda w.\ z)$.
Primero, la evaluemos en orden normal:
\begin{equation*}
  \begin{aligned}
    &(\lambda f.\lambda x.\ f(fx))(\lambda z.\lambda x.\lambda y.\ zyx)(\lambda z.\lambda w.\ z) \Rightarrow_N \\ 
    &\quad\left[
      \begin{aligned}
        &(\lambda f.\lambda x.\ f(fx))(\lambda z.\lambda x.\lambda y.\ zyx) \Rightarrow_N \\ 
          &\quad\left[\lambda f.\lambda x.\ f(fx) \Rightarrow_N \lambda f.\lambda x.\ f(fx)\right. \\ 
          &\quad\left[
            \begin{aligned}
              &((\lambda x.\ f(fx))/f \to (\lambda z.\lambda x.\lambda y.\ zyx)) \Rightarrow_N \\
              &\Rightarrow_N \lambda w.\ (\lambda z.\lambda x.\lambda y.\ zyx)((\lambda z.\lambda x.\lambda y.\ zyx)w) \\
            \end{aligned}
          \right. \\ 
        &\Rightarrow_N \lambda w.\ (\lambda z.\lambda x.\lambda y.\ zyx)((\lambda z.\lambda x.\lambda y.\ zyx)w) \\
      \end{aligned}
    \right. \\
    &\quad\left[
      \begin{aligned}
        &(((\lambda z.\lambda x.\lambda y.\ zyx)((\lambda z.\lambda x.\lambda y.\ zyx)w)) / w \to \lambda z.\lambda w.\ z) \Rightarrow_N \\ 
        &\Rightarrow_N (\lambda z.\lambda x.\lambda y.\ zyx)((\lambda z.\lambda x.\lambda y.\ zyx)(\lambda z.\lambda w.\ z)) \\ 
          &\quad\left[\lambda z.\lambda x.\lambda y.\ zyx \Rightarrow_N \lambda z.\lambda x.\lambda y.\ zyx\right. \\ 
          &\quad\left[
            \begin{aligned}
              &((\lambda x.\lambda y.\ zyx) / z \to ((\lambda z.\lambda x.\lambda y.\ zyx)(\lambda z.\lambda w.\ z))) \Rightarrow_N \\ 
              &\Rightarrow_N \lambda s.\lambda t.\ (\lambda z.\lambda x.\lambda y.\ zyx)(\lambda z.\lambda w.\ z)yx
            \end{aligned}
          \right. \\ 
        &\Rightarrow_N \lambda s.\lambda t.\ (\lambda z.\lambda x.\lambda y.\ zyx)(\lambda z.\lambda w.\ z)yx \\
      \end{aligned}
    \right. \\ 
    &\Rightarrow_N \lambda s.\lambda t.\ (\lambda z.\lambda x.\lambda y.\ zyx)(\lambda z.\lambda w.\ z)yx
  \end{aligned}
\end{equation*}

Y ahora la evaluemos en orden eager:
\begin{equation*}
  \begin{aligned}
    &(\lambda f.\lambda x.\ f(fx))(\lambda z.\lambda x.\lambda y.\ zyx)(\lambda z.\lambda w.\ z) \Rightarrow_E \\
    &\quad\left[
      \begin{aligned}
        &(\lambda f.\lambda x.\ f(fx))(\lambda z.\lambda x.\ zyx) \Rightarrow_E \\ 
        &\quad\left[
          \begin{aligned}
            &\lambda f.\lambda x.\ f(fx) \Rightarrow_E \lambda f.\lambda x.\ f(fx)
          \end{aligned}
        \right. \\ 
        &\quad\left[
          \begin{aligned}
            &\lambda z.\lambda x.\lambda y.\ zyx \Rightarrow_E \lambda z.\lambda x.\lambda y.\ zyx
          \end{aligned}
        \right. \\ 
        &\quad\left[
          \begin{aligned}
            &((\lambda x.\ f(fx)) / f \to (\lambda z.\lambda x.\ zyx)) \Rightarrow_E \\ 
            &\Rightarrow_E \lambda t.\ (\lambda z.\lambda x.\lambda y.\ zyx)((\lambda z.\lambda x.\lambda y.\ zyx)t)
          \end{aligned}
        \right. \\ 
        &\Rightarrow_E \lambda t.\ (\lambda z.\lambda x.\lambda y.\ zyx)((\lambda z.\lambda x.\lambda y.\ zyx)t)
      \end{aligned}
    \right. \\ 
    &\quad\left[
      \begin{aligned}
        &\lambda z.\lambda w.\ z \Rightarrow_E \lambda z.\lambda w.\ z
      \end{aligned}
    \right. \\ 
    &\quad\left[
      \begin{aligned}
        &(((\lambda z.\lambda x.\lambda y.\ zyx)((\lambda z.\lambda x.\lambda y.\ zyx)t)) / t \to (\lambda z.\lambda w.\ z)) \Rightarrow_E \\ 
        &\Rightarrow_E (\lambda z.\lambda x.\lambda y.\ zyx)((\lambda z.\lambda x.\lambda y.\ zyx)(\lambda t.\lambda w.\ t)) \\ 
        &\quad\left[
          \begin{aligned}
            &\lambda z.\lambda x.\lambda y.\ zyx \Rightarrow_E \dots
          \end{aligned}
        \right. \\ 
        &\quad\left[
          \begin{aligned}
            &(\lambda z.\lambda x.\lambda y.\ zyx)(\lambda t.\lambda w.\ t) \Rightarrow_E \\ 
            &\quad\left[
              \begin{aligned}
                &\lambda z.\lambda x.\lambda y.\ zyx \Rightarrow_E \dots
              \end{aligned}
            \right. \\ 
            &\quad\left[
              \begin{aligned}
                &\lambda t.\lambda w.\ t \Rightarrow_E \dots
              \end{aligned}
            \right. \\ 
            &\quad\left[
              \begin{aligned}
                &(\lambda x.\lambda y.\ zyx) / z \to (\lambda t.\lambda w.\ t) \Rightarrow_E \\ 
                &\Rightarrow_E \lambda x.\lambda y.\ (\lambda t.\lambda w.\ t)yx
              \end{aligned}
            \right. \\ 
            &\Rightarrow_E \lambda x.\lambda y.\ (\lambda t.\lambda w.\ t)yx
          \end{aligned}
        \right. \\ 
        &\quad\left[
          \begin{aligned}
            &(\lambda x.\lambda y.\ zyx) / z \to (\lambda x.\lambda y.\ (\lambda t.\lambda w.\ t)yx) \Rightarrow_E \\
            &\Rightarrow_E \lambda x.\lambda y.\ (\lambda s.\lambda r.\ (\lambda t.\lambda w.\ t)rs)yx
          \end{aligned}
        \right. \\ 
        &\Rightarrow_E \lambda x.\lambda y.\ (\lambda s.\lambda r.\ (\lambda t.\lambda w.\ t)rs)yx
      \end{aligned}
    \right. \\ 
    &\Rightarrow_E \lambda x.\lambda y.\ (\lambda s.\lambda r.\ (\lambda t.\lambda w.\ t)rs)yx 
  \end{aligned}
\end{equation*}

\subsection*{Item B}
Tenemos la expresión $(\lambda z.\ zz)(\lambda f.\lambda x.\ f(fx))$.
Primero la evaluemos en orden normal:
\begin{equation*}
  \begin{aligned}
    &(\lambda z.\ zz)(\lambda f.\lambda x.\ f(fx)) \Rightarrow_N \\
    &\quad\left[
      \begin{aligned}
        &\lambda z.\ zz \Rightarrow_N \dots
      \end{aligned}
    \right. \\ 
    &\quad\left[
      \begin{aligned}
        &(zz) / z \to (\lambda f.\lambda x.\ f(fx)) \Rightarrow_N \\ 
        &\Rightarrow_N (\lambda f.\lambda x.\ f(fx))(\lambda f.\lambda x.\ f(fx)) \\
        &\quad\left[
          \begin{aligned}
            &\lambda f.\lambda x.\ f(fx) \Rightarrow_N \dots
          \end{aligned}
        \right. \\ 
        &\quad\left[
          \begin{aligned}
            &(\lambda x.\ f(fx)) / f \to (\lambda f.\lambda x.\ f(fx)) \Rightarrow_N \\ 
            &\lambda y.\ (\lambda f.\lambda x.\ f(fx))((\lambda f.\lambda x.\ f(fx))y)
          \end{aligned}
        \right. \\ 
        &\Rightarrow_N \lambda y.\ (\lambda f.\lambda x.\ f(fx))((\lambda f.\lambda x.\ f(fx))y)
      \end{aligned}
    \right. \\ 
    &\Rightarrow_N \lambda y.\ (\lambda f.\lambda x.\ f(fx))((\lambda f.\lambda x.\ f(fx))y)
  \end{aligned}
\end{equation*}

Y en orden eager tenemos:
\begin{equation*}
  \begin{aligned}
    &(\lambda z.\ zz)(\lambda f.\lambda x.\ f(fx)) \Rightarrow_E \\
    &\quad\left[
      \begin{aligned}
        &\lambda z.\ zz \Rightarrow_E \dots
      \end{aligned}
    \right. \\ 
    &\quad\left[
      \begin{aligned}
        &\lambda f.\lambda x.\ f(fx) \Rightarrow_E \dots
      \end{aligned}
    \right. \\ 
    &\quad\left[
      \begin{aligned}
        &(zz) / z \to (\lambda f.\lambda x.\ f(fx)) \Rightarrow_E \\
        &\Rightarrow_E (\lambda f.\lambda x.\ f(fx))(\lambda f.\lambda x.\ f(fx)) \\
        &\quad\left[
          \begin{aligned}
            &\lambda f.\lambda x.\ f(fx) \Rightarrow_E \dots
          \end{aligned}
        \right. \\ 
        &\quad\left[
          \begin{aligned}
            &\lambda f.\lambda x.\ f(fx) \Rightarrow_E \dots
          \end{aligned}
        \right. \\ 
        &\quad\left[
          \begin{aligned}
            &(\lambda x.\ f(fx)) / f \to (\lambda f.\lambda x.\ f(fx)) \Rightarrow_E \\ 
            &\Rightarrow_E \lambda y.\ (\lambda f.\lambda x.\ f(fx))((\lambda f.\lambda x.\ f(fx))y) 
          \end{aligned}
        \right. \\ 
        &\Rightarrow_E \lambda y.\ (\lambda f.\lambda x.\ f(fx))((\lambda f.\lambda x.\ f(fx))y)
      \end{aligned}
    \right. \\ 
    &\Rightarrow_E \lambda y.\ (\lambda f.\lambda x.\ f(fx))((\lambda f.\lambda x.\ f(fx))y)
  \end{aligned}
\end{equation*}

\section*{Ejercicio 6}
El \textbf{item A} lo paso para poder completar varios ejercicios para el parcial.

Respecto al \textbf{item B}, es verdadero por el punto (a) y luego el Teorema de Church-Rosser.

\section*{Ejercicio 7}
No es cierto que $NOT\ TRUE \Rightarrow FALSE$ porque se llega a una forma canónica antes.
En particular, se puede notar que:
\begin{equation*}
  \begin{aligned}
    NOT\ TRUE &\equiv (\lambda b.\lambda x.\lambda y.\ byx)(\lambda x.\lambda y.\ x) \\ 
              &\Rightarrow \lambda x.\lambda y.\ (\lambda s.\lambda t.\ s)yx
  \end{aligned}
\end{equation*}
para ambos órdenes.

\section*{Ejercicio 8}
Suponiendo que $e_0$ y $e_1$ son formas canónicas, podemos ver la evaluación de cada una de las expresiones por separado.

\subsection*{Item A}
\begin{equation*}
  \begin{aligned}
    &NOT\ TRUE\ e_0 e_1 \Rightarrow_E \\ 
    &\quad\left[
      \begin{aligned}
        &NOT\ TRUE\ e_0 \Rightarrow_E \\
        &\quad\left[
          \begin{aligned}
            &NOT\ TRUE \equiv (\lambda b.\lambda x.\lambda y.\ byx)(\lambda x.\lambda y.\ x) \Rightarrow_E \\ 
            &\quad\left[
              \begin{aligned}
                &\lambda b.\lambda x.\lambda y.\ byx \Rightarrow_E \dots
              \end{aligned}
            \right. \\
            &\quad\left[
              \begin{aligned}
                &\lambda x.\lambda y.\ x \Rightarrow_E \dots
              \end{aligned}
            \right. \\
            &\quad\left[
              \begin{aligned}
                &(\lambda x.\lambda y.\ byx) / b \to (\lambda x.\lambda y.\ x) \equiv \lambda x.\lambda y.\ (\lambda r.\lambda s.\ r)yx \Rightarrow_E \dots
              \end{aligned}
            \right. \\ 
            &\Rightarrow_E \lambda x.\lambda y.\ (\lambda r.\lambda s.\ r)yx 
          \end{aligned}
        \right. \\
        &\quad\left[
          \begin{aligned}
            &e_0 \Rightarrow_E \dots
          \end{aligned}
        \right. \\
        &\quad\left[
          \begin{aligned}
            &(\lambda y.\ (\lambda r.\lambda s.\ r)yx) / x \to e_0 \equiv \lambda y.\ (\lambda r.\lambda s.\ r)ye_0 \Rightarrow_E \dots
          \end{aligned}
        \right. \\ 
        &\Rightarrow_E \lambda y.\ (\lambda r.\lambda s.\ r)ye_0 
      \end{aligned}
    \right. \\
    &\quad\left[
      \begin{aligned}
        &e_1 \Rightarrow_E \dots
      \end{aligned}
    \right. \\ 
    &\quad\left[
      \begin{aligned}
        &(\lambda r.\lambda s.\ r)ye_0 / y \to e_1 \equiv (\lambda r.\lambda s.\ r) e_1 e_0 \Rightarrow_E \\ 
        &\quad\left[
          \begin{aligned}
            &(\lambda r.\lambda s.\ r) e_1 \Rightarrow_E \\
            &\quad\left[
              \begin{aligned}
                &\lambda r.\lambda s.\ r \Rightarrow_E \dots
              \end{aligned}
            \right. \\ 
            &\quad\left[
              \begin{aligned}
                &e_1 \Rightarrow_E \dots
              \end{aligned}
            \right. \\ 
            &\quad\left[
              \begin{aligned}
                &(\lambda s.\ r) / r \to e_1 \equiv \lambda s.\ e_1 \Rightarrow_E \dots
              \end{aligned}
            \right. \\ 
            &\Rightarrow_E \lambda s.\ e_1
          \end{aligned}
        \right. \\ 
        &\quad\left[
          \begin{aligned}
            &e_0 \Rightarrow_E \dots
          \end{aligned}
        \right. \\ 
        &\quad\left[
          \begin{aligned}
            &e_1 / s \to e_0 \equiv e_1 \Rightarrow_E \dots
          \end{aligned}
        \right. \\ 
        &\Rightarrow_E e_1
      \end{aligned}
    \right. \\ 
    &\Rightarrow_E e_1
  \end{aligned}
\end{equation*}

\subsection*{Item B}
\begin{equation*}
  \begin{aligned}
    &AND\ FALSE\ (\Delta\Delta) e_0 e_1 \Rightarrow_N \\ 
    &\quad\left[
      \begin{aligned}
        &AND\ FALSE\ (\Delta\Delta) e_0 \Rightarrow_N \\
        &\quad\left[
          \begin{aligned}
            &AND\ FALSE\ (\Delta\Delta) \Rightarrow_N \\
            &\quad\left[
              \begin{aligned}
                &AND\ FALSE \Rightarrow_N \\
                &\quad\left[
                  \begin{aligned}
                    &AND \equiv \lambda b.\lambda c.\lambda x.\lambda y.\ b(cxy)y \Rightarrow_N \dots
                  \end{aligned}
                \right. \\ 
                &\quad\left[
                  \begin{aligned}
                    &(\lambda c.\lambda x.\lambda y.\ b(cxy)y) / b \to FALSE \equiv \lambda c.\lambda x.\lambda y.\ FALSE(cxy)y \Rightarrow_N \dots
                  \end{aligned}
                \right. \\ 
                &\Rightarrow_N \lambda c.\lambda x.\lambda y.\ FALSE(cxy)y
              \end{aligned}
            \right. \\ 
            &\quad\left[
              \begin{aligned}
                &(\lambda x.\lambda y. FALSE(cxy)y) / c \to (\Delta\Delta) \equiv \lambda x.\lambda y.\ FALSE((\Delta\Delta)xy)y \Rightarrow_N \dots
              \end{aligned}
            \right. \\ 
            &\Rightarrow_N \lambda x.\lambda y.\ FALSE((\Delta\Delta)xy)y 
          \end{aligned}
        \right. \\ 
        &\quad\left[
          \begin{aligned}
            &(\lambda y.\ FALSE((\Delta\Delta)xy)y) / x \to e_0 \equiv \lambda y.\ FALSE((\Delta\Delta)e_0y)y \Rightarrow_N \dots
          \end{aligned}
        \right. \\ 
        &\Rightarrow_N \lambda y.\ FALSE((\Delta\Delta)e_0y)y
      \end{aligned}
    \right. \\ 
    &\quad\left[
      \begin{aligned}
        &(FALSE((\Delta\Delta)e_0y)y) / y \to e_1 \equiv FALSE((\Delta\Delta)e_0e_1)e_1 \Rightarrow_N \\
        &\quad\left[
          \begin{aligned}
            &FALSE((\Delta\Delta)e_0e_1) \equiv (\lambda x.\lambda y.\ y)((\Delta\Delta)e_0e_1) \Rightarrow_N \\ 
            &\quad\left[
              \begin{aligned}
                &\lambda x.\lambda y.\ y \Rightarrow_N \dots
              \end{aligned}
            \right. \\ 
            &\quad\left[
              \begin{aligned}
                &(\lambda y.\ y) / x \to ((\Delta\Delta)e_0e_1) \equiv \lambda y.\ y \Rightarrow_N \dots
              \end{aligned}
            \right. \\ 
            &\Rightarrow_N \lambda y.\ y
          \end{aligned}
        \right. \\ 
        &\quad\left[
          \begin{aligned}
            &(y) / y \to e_1 \equiv e_1
          \end{aligned}
        \right. \\ 
        &\Rightarrow_N e_1
      \end{aligned}
    \right. \\ 
    &\Rightarrow_N e_1
  \end{aligned}
\end{equation*}

\section*{Ejercicio 9}
Ninguna tiene forma canónica y es debido a que el orden eager implica que se debe evaluar el argumento.
Luego, en algún momento se debe evaluar $\Delta\Delta$ en el primer caso y $\lambda w.\ \Delta\Delta$ en el segundo, las cuales no tienen forma canónica (i.e., no terminan).

\section*{Ejercicio 10}
Me lo salto porque interpreto que al no ser una abstracción sino una aplicación, entonces no se puede aplicar la contracción $\eta$.

\end{document}
